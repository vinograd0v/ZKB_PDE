%!TEX root = main.tex
Consideremos el problema de valor inicial asociado a la \textit{ecuación con no linealidad modificada de Zakharov-Kusnetsov-Burgers}
\begin{equation}\label{ZKB}
    \begin{cases}
    u_t+\partial_{x_1}\Delta u-\Delta u+u^3=0, & (x,t)\in(-\pi,\pi)^n\times(0,\infty),\\
    u(x,0)=u_0(x), & x\in[-\pi,\pi]^n.
\end{cases}
\end{equation}
Procedamos de manera formal en búsqueda de un candidato a solución, tomando la transformada de Fourier respecto a la variable espacial
\begin{align*}
   (u_t+\partial_{x_1}\Delta u-\Delta u+u^3)^{\wedge}(k)&= \widehat{u}_t(k)+\widehat{\partial_{x_1}\Delta u}(k)-\widehat{\Delta u}(k)+\widehat{u^3}(k)\\
   &=\partial_t\widehat{u}(k)+ik_1\widehat{\Delta u}(k)-\widehat{\Delta u}(k)+\widehat{u^3}(k)\\
   &=\partial_t\widehat{u}(k)+(ik_1-1)\sum_{i=1}^n\widehat{\partial^2_{x_i}u}(k)+\widehat{u^3}(k)\\
   &=\partial_t\widehat{u}(k)+(ik_1-1)\sum_{i=1}^ni^2k_i^2\widehat{u}(k)+\widehat{u^3}(k)\\
   &=\partial_t\widehat{u}(k)+(1-ik_1)|k|^2\widehat{u}(k)+\widehat{u^3}(k).\\
\end{align*}
Así junto al hecho de que $\widehat{u}(k,0)=\widehat{u}_0(k)$ para todo $k\in\Z^n$ tenemos una ecuación diferencial ordinaria asociada a un problema de valor inicial respecto a la variable temporal 
$$\begin{cases}
    \dfrac{d}{dt}\widehat{u}(k)+(|k|^2-ik_1|k|^2)\widehat{u}(k)=-\widehat{u^3}(k), & k\in\Z^n,t>0\\
    \widehat{u}(k,0)=\widehat{u}_0(k), &k\in\Z.
\end{cases}$$
Luego usando el factor integrante $e^{(|k|^2-ik_1|k|^2)t}$, e integrando a ambos lados de 0 a $t$ tenemos que
$$e^{(|k|^2-ik_1|k|^2)t}\widehat{u}(k,t)-\widehat{u}_0(k)=-\int_0^te^{(|k|^2-ik_1|k|^2)t^\prime}\widehat{u^3}(k,t^\prime)\,dt^\prime.$$
Así despejando $\widehat{u}(k,t)$ llegamos a que
$$\widehat{u}(k,t)=e^{(ik_1|k|^2-|k|^2)t}\widehat{u}_0(k)-\int_0^te^{(ik_1|k|^2-|k^2|)(t-t^\prime)}\widehat{u^3}(k,t^\prime)\,dt^\prime,$$
tomando la transformada inversa de Fourier tenemos que
\begin{align*}
    u(x,t)&=(e^{(ik_1|k|^2-|k^2|)t}\widehat{u}_0(k))^{\vee}-\int_0^t(e^{(ik_1|k|^2-|k^2|)(t-t^\prime)}\widehat{u^3}(k,t^\prime))^{\vee}\,dt^\prime\\
    &=\sum_{k\in\Z^n}e^{(ik_1|k|^2-|k^2|)t+ik\cdot x}\widehat{u}_0(k)-\int_0^t\sum_{k\in\Z^n}e^{(ik_1|k|^2-|k^2|)(t-t^\prime)+ik\cdot x}\widehat{u^3}(k,t^\prime)\,dt^\prime.
\end{align*}
Teniendo en cuenta esto, es en cierta medida natural definir la familia de operadores $\{U(t)\}_{t\geq 0}$ tal que:
$$U(t)f(x)=\begin{cases}
    \displaystyle\sum_{k\in\Z^n}e^{(ik_1|k|^2-|k^2|)t+ik\cdot x}\widehat{f}(k), & t>0,\\
    f(x), &t=0
\end{cases}$$
Para $f$ lo suficientemente regular. A continuación presentamos una propiedad de estos operadores:
\begin{prop}\label{FourierUt}
    Si $f\in L^2(\T^n)$ entonces $U(t)f\in L^2(\T^n),$ ademas
    $$\hat{U(t)f}(k)=e^{(ik_1|k|^2-|k^2|)t}\hat{f}(k).$$
    Para todo $k\in \Z^n.$ 
 \end{prop} 


De esta manera, nuestro candidato a solución es la formula de Duhamel dada por
\begin{equation}\label{Duhamel}
u(x,t)=U(t)u_0(x)-\int_0^tU(t-t^\prime)u^3(x,t^\prime)\,dt^\prime,
\end{equation}
en algunas ocasiones omitiremos la dependencia espacial con el propósito de no sobrecargar la notación.
\begin{definition}
    Dado $s\geq 0.$ Diremos que el problema (\ref{ZKB}) esta localmente bien planteado en $H^s(\T^n)$ si:
    \begin{itemize}
        \item \textbf{(Existencia y Unicidad).} Dado $u_0\in H^s(\T^n),$ existe $T>0$ y una única solución de la formula de Duhamel (\ref{Duhamel}) con dato inicial $u_0$ en el espacio $C([0,T];H^s(\T^n)).$
        \item \textbf{(Dependencia Continua).} Dado $u_0\in H^s(\T^n)$ existe una vecindad $V$ de $u_0$ en $H^s(\T^n)$ y $T>0$ tal que la aplicación dato inicial solución 
        $$v_0\in H^s(\T^n)\mapsto v\in C([0,T];H^s(\T^n)). $$ 
        Es continua.
    \end{itemize}
\end{definition}
Antes de continuar con el resultado principal debemos hacer varias salvedades. Primero definimos el espacio $C([0,T];H^s(\T^n))$.
\begin{definition}
     El espacio $C([0,T];H^s(\T^n))$ consiste de funciones $f:[0,T]\to H^s(\T^n)$, continuas en el siguiente sentido:\\ 
 Dado $t^\prime\in(0,T)$ tenemos que
$$\lim_{t\to t^\prime}\|f(t)-f(t^\prime)\|_{H^s}=0,$$
si $t^\prime=0$ o $t^\prime=T$ se toma el limite lateral.
 \end{definition}
Una propiedad importante de este espacio es
\begin{prop}\label{completezC}
    El espacio $C([0,T];H^s(\T^n))$ es completo con la distancia
    $$\sup_{t\in[0,T]}\|u(t)-v(t)\|_{H^s}.$$
    Para $u,v\in C([0,T];H^s(\T^n)).$
\end{prop}
\begin{proof}
    \textcolor{blue}{luego escribo lo que dijo oscar}
\end{proof}
Segundo introducimos los conceptos claves de contracción y punto fijo.  
\begin{definition}Sea $(M,d)$ un espacio métrico y $\psi:M\to M$ una función.
    \begin{itemize}
        \item $\psi$ es una \textbf{contracción} si existe $0<L<1$ tal que
        $$d(\psi(x),\psi(y))\leq Ld(x,y),$$
        para todo $x,y\in M.$
        \item Dado $x\in M,$ si $\psi(x)=x$, se dice que $x$ es un \textbf{punto fijo} de $\psi.$
    \end{itemize}
\end{definition}
\begin{theorem}[Punto fijo de Banach]\label{Banach}
Sea $(M,d)$ un espacio métrico completo. entonces toda contracción tiene un único punto fijo.   
\end{theorem}
La idea para el buen planteamiento es ver (\ref{Duhamel}) como una aplicación y usar (\ref{Banach}) para encontrar soluciones como puntos fijos en espacios de funciones adecuados. Con todo esto procedemos a la demostración de el resultado principal, la buena colocación local.
\begin{theorem}
    Para cualquier $n\geq 1$ fijo, el problema de Cauchy (\ref{ZKB}) esta localmente bien planteado en $H^s(\T^n),$ $s>\frac{n}{2}.$ Esto es:
    \begin{itemize}
        \item \textbf{(Existencia y Unicidad).} Para cualquier $u_0\in H^s(\T^n),$ existe un tiempo $T>0$ y una única $u\in C([0,T];H^s(\T^n))$ solución de la formula integral (\ref{Duhamel}) con dato inicial $u_0.$
        \item \textbf{(Dependencia Continua).} Dado $u_0\in H^s(\T^n)$ existe una vecindad $V$ de $u_0$ en $H^s(\T^n)$ y $T>0$ tal que la aplicación dato inicial solución 
        $$v_0\in V\mapsto v\in C([0,T];H^s(\T^n)). $$ 
        Es continua. 
    \end{itemize}
\end{theorem}
\begin{proof}
Dividiremos la prueba en tres partes.
\begin{itemize}
       \item[i)]\textbf{Existencia.} Sea $u_0\in H^s(\T^n)$ con $s>\frac{n}{2}$ arbitrario pero fijo. Veamos que existe una solución de (\ref{Duhamel}) con dato inicial $u_0.$ Si $u_0=0,$ tomando $u=0$ tenemos la existencia. Ahora si asumimos $u_0\neq 0$ tenemos que $\|u_0\|_{H^s}>0.$ Dados $T>0$ y $a>0$ fijos definimos el espacio
       $$X_T^s(a)=\{v\in C([0,T];H^s(\T^n)):\sup_{t\in[0,T]}\|v(t)\|_{H^s}\leq a\}.$$
       Primero probemos que $X_T^s(a)$ es completo con la distancia
       $$\sup_{t\in[0,T]}\|u(t)-v(t)\|_{H^s},$$
       para $u,v\in X^s_T(a)$.\\

       Note que $X^s_T(a)$ es un subconjunto cerrado de $C([0,T];H^s(\T^n))$, ya que es la bola cerrada de centro $0$ en ese espacio. Por la proposición (\ref{completezC}) y el hecho de que todo subconjunto cerrado de un espacio métrico completo es completo, concluimos que $X^s_T(a)$ es completo bajo esa distancia.

       Con esto, sea $v\in X^s_T(a)$, definimos la función
       $$\phi(v)(x,t)=U(t)u_0(x)-\int_0^tU(t-t^\prime)v^3(x,t^\prime)\,dt^\prime.$$
       Veamos que $\phi(v)\in X^s_T(a).$ 

       Utilizando la definición de la norma en $H^s(\T^n)$ y la proposición (\ref{FourierUt}) tenemos que
       \begin{align*}
           \|\phi(v)(x,t)\|_{H^s}&\leq\|U(t)u_0(x)\|_{H_s}+\int_0^t\|U(t-t^\prime)v^3(x,t^\prime)\|_{H^s}\,dt^\prime\\
           &=\|\langle k\rangle^s\widehat{U(t)u}_0(k)\|_2+\int_0^t\|\langle k\rangle^s(U(t-t^\prime)v^3)^\wedge(k,t^\prime)\|_2\,dt^\prime\\
           &=\|e^{(ik_1|k|^2-|k|^2)t}\langle k\rangle^s\widehat{u}_0(k)\|_2+\int_0^t\|e^{(ik_1|k|^2-|k|^2)(t-t^\prime)}\langle k\rangle^s\widehat{v^3}(k,t^\prime)\|_2\,dt^\prime\\
           &\leq\|u_0(x)\|_{H^s}+\int_0^t\|v^3(x,t^\prime)\|_{H^s}\,dt^\prime\\
           &\leq\|u_0(x)\|_{H^s}+c\int_0^t\|v(x,t^\prime)\|^3_{H^s}\,dt^\prime.
       \end{align*}
       Note que esto lo podemos hacer ya que $H^s(\T^n)$ es álgebra de Banach y 
       $$\left|e^{(ik_1|k|^2-|k|^2)t}\right|=\left|e^{ik_1|k|^2t}\right|\left|e^{-|k|^2t}\right|\leq 1.$$
       Si tomamos el supremo a ambos lados de la desigualdad tenemos que
       \begin{align*}
           \sup_{t\in[0,T]}\|\phi(v)(x,t)\|_{H^s}&\leq\|u_0(x)\|_{H^s}+cT\sup_{t\in[0,T]}\|v(x,t)\|^3\\
           &\leq\|u_0(x)\|_{H^s}+cTa^3,
       \end{align*}
       esto ultimo ya que $v\in X_T^s(a).$ Note que si escogemos
       $$\begin{cases}
           a=\dfrac{3}{2}\|u_0(x)\|_{H^s},\\
           0<T<\dfrac{4}{27c}\|u_0(x)\|^{-2}_{H^s},
       \end{cases}$$
       tenemos que
       $$\sup_{t\in[0,T]}\|\phi(v)(x,t)\|_{H^s}<\|u_0(x)\|_{H^s}+c\left(\dfrac{4}{27c}\|u_0(x)\|^{-2}_{H^s}\right)\left(\dfrac{3}{2}\|u_0(x)\|_{H^s}\right)^3 =\dfrac{3}{2}\|u_0(x)\|_{H^s}.$$
       Ahora falta probar que $\phi(v)\in C([0,T];H^s(\T^n))$. Primero notemos que dado $s\in[0,T]$
       \begin{align*}
           \|\phi(v)(t)-\phi(v)(s)\|_{H^s}&=\left\|U(t)u_0-\int_0^tU(t-t^\prime)v^3(t^\prime)\,dt^\prime-U(s)u_0+\int_0^sU(s-t^\prime)v^3(t^\prime)\,dt^\prime\right\|_{H^s}\\
           &\leq\|U(t)u_0-U(s)u_0\|_{H^s}+\left\|\int_0^sU(s-t^\prime)v^3(t^\prime)\,dt^\prime-\int_0^tU(t-t^\prime)v^3(t^\prime)\,dt^\prime\right\|_{H^s}.
       \end{align*}
       Luego basta con probar que cuando $t\to s$, cada sumando se hace 0. Primero veamos el caso donde $s=0$, note que la definición de la norma en $H^s(\T^n)$, y por la proposición (\ref{FourierUt}) 
       \begin{align*}
           \|U(t)u_0-U(0)u_0\|_{H^s}^2&=\|\langle k\rangle^s(\hat{U(t)u}_0(k)-\hat{u_0}(k))\|_2^2\\
           &=\sum_{k\in\Z^n}|\langle k\rangle^s(e^{(ik_1|k|^2-|k|^2)t}-1)\hat{u}_0(k)|^2.
       \end{align*}
       Ahora note que
       \begin{align*}
           |\langle k\rangle^s(e^{(ik_1|k|^2-|k|^2)t}-1)\hat{u}_0(k)|^2&\leq(|e^{(ik_1|k|^2-|k|^2)t}|+|1|)^2||\langle k\rangle^s\hat{u}_0(k)|^2\\
           &\leq 4|\langle k\rangle^s\hat{u}_0(k)|^2,
       \end{align*}
        por hipótesis $\{\langle k\rangle^s\hat{u}_0(k)\}\in\ell^2(\Z^n)$, por lo tanto  $\{\langle k\rangle^s(e^{(ik_1|k|^2-|k|^2)t}-1)\hat{u}_0(k)\}\in\ell^2(\Z^n)$, por M de Weierstrass. Así tenemos que
       $$\lim_{t\to0^+}\|U(t)u_0-U(0)u_0\|_{H^s}=\left(\sum_{k\in\Z^n}\lim_{t\to0^+}|\langle k\rangle^s(e^{(ik_1|k|^2-|k|^2)t}-1)\hat{u}_0(k)|^2\right)^{\frac{1}{2}}=0.$$
       Ahora para el otro sumando tenemos que
       
        \begin{align*}
             \left\|\int_0^sU(s-t^\prime)v^3(t^\prime)\,dt^\prime-\int_0^tU(t-t^\prime)v^3(t^\prime)\,dt^\prime\right\|_{H^s}&\leq\int_0^t\|U(t-t^\prime)v^3(t^\prime)\|_{H^s}\,dt^\prime\\
             &\leq\int_0^t\|v^3(t^\prime)\|_{H^s}\,dt^\prime\to 0
         \end{align*} 

       Cuando $t\to0^+$, de esta manera concluimos que

       $$\lim_{t\to0^+}\|\phi(v)(t)-\phi(v)(0)\|_{H^s}=0.$$

       Ahora, si $s\neq 0$, podemos asumir sin perdida de generalidad que $t<s$, de manera similar para el primer sumando
       \begin{align*}
           \|U(t)u_0-U(s)u_0\|_{H^s}^2&=\|\langle k\rangle^s(\hat{U(t)u}_0(k)-\hat{U(s)u_0}(k))\|_2^2\\
           &=\sum_{k\in\Z^n}|\langle k\rangle^s(e^{(ik_1|k|^2-|k|^2)t}-e^{(ik_1|k|^2-|k|^2)s})\hat{u}_0(k)|^2.
       \end{align*}
       Luego
       \begin{align*}
           |\langle k\rangle^s(e^{(ik_1|k|^2-|k|^2)t}-e^{(ik_1|k|^2-|k|^2)s})\hat{u}_0(k)|^2&\leq(|e^{(ik_1|k|^2-|k|^2)t}|+|e^{(ik_1|k|^2-|k|^2)s}|)^2||\langle k\rangle^s\hat{u}_0(k)|^2\\
           &\leq 4|\langle k\rangle^s\hat{u}_0(k)|^2,
       \end{align*}
       y nuevamente por M de Weierstrass podemos ingresar el limite a la suma, es decir
       $$\lim_{t\to s}\|U(t)u_0-U(s)u_0\|_{H^s}=\left(\sum_{k\in\Z^n}\lim_{t\to s}|\langle k\rangle^s(e^{(ik_1|k|^2-|k|^2)t}-e^{(ik_1|k|^2-|k|^2)s})\hat{u}_0(k)|^2\right)^{\frac{1}{2}}=0.$$
       Para el caso del sumando con la integral tenemos

       \begin{align*}
             \left\|\int_0^sU(s-t^\prime)v^3(t^\prime)\,dt^\prime-\int_0^tU(t-t^\prime)v^3(t^\prime)\,dt^\prime\right\|_{H^s}&\leq\int_t^s\|U(s-t^\prime)v^3(t^\prime)-U(t-t^\prime)v^3(t^\prime)\|_{H^s}\,dt^\prime\\
             &\leq\int_t^s\|U(s-t^\prime)v^3(t^\prime)\|_{H^s}+\|U(t-t^\prime)v^3(t^\prime)\|_{H^s}\,dt^\prime\\
             &\leq2\int_t^s\|v^3(t^\prime)\|_{H^s}dt^\prime\to 0.
         \end{align*} 
         Cuando $t\to s$. Es decir
         $$\lim_{t\to s}\|\phi(v)(t)-\phi(v)(s)\|_{H^s}=0.$$
       Con estos dos hechos podemos concluir que $\phi:X^s_T(a)\to X^s_T(a).$ Ahora veamos que $\phi$ define una contracción para los $a$ y $T$ dados previamente. Sean $u,v\in X^s_T(a),$ Note que por la definición de $\phi$ y de manera similar al argumento previo tenemos
       \begin{align*}
           \|\phi(v)(t)-\phi(u)(t)\|_{H^s}&=\left\|-\int_0^tU(t-t^\prime)v^3(t^\prime)\,dt^\prime+\int_0^tU(t-t^\prime)u^3(t^\prime)\,dt^\prime\right\|_{H^s}\\
           &\leq\int_0^t\|U(t-t^\prime)(u^3(t^\prime)-v^3(t^\prime))\|_{H^s}\,dt^\prime\\
           &\leq\int_0^t\|u^3(t^\prime)-v^3(t^\prime)\|_{H^s}\,dt^\prime\\
           &=\int_0^t\|(u-v)(u^2+uv+v^2)(t^\prime)\|_{H^s}\,dt^\prime\\
           &\leq c\int_0^t\|u(t^\prime)-v(t^\prime)\|_{H^s}\|(u^2+uv+v^2)(t^\prime)\|_{H^s}\,dt^\prime\\
           &\leq c\int_0^t\|u(t^\prime)-v(t^\prime)\|_{H^s}(\|u^2(t^\prime)\|_{H^s}+\|uv(t^\prime)\|_{H^s}+\|v^2(t^\prime)\|_{H^s})\,dt^\prime.
       \end{align*}
        Con esto si tomamos el supremo a ambos lados respecto a $t$ tenemos que
       $$\sup_{t\in[0,T]}\|\phi(v)(x,t)-\phi(u)(x,t)\|_{H^s}\leq 3ca^2T\sup\|u(x,t)-v(x,t)\|_{H^s}.$$
       Luego note que
       $$0<3ca^2T<3c\left(\dfrac{3}{2}\|u_0(x)\|_{H^s}\right)^2\left(\dfrac{4}{27c}\|u_0(x)\|^{-2}_{H^s}\right)=1$$
       De esta manera concluimos que $\phi$ es una contracción sobre un espacio métrico completo, y por el teorema (\ref{Banach}) tenemos que existe una única función $u\in X^s_T(a)$ donde $\phi(u)=u,$ esto quiere decir que $u$ es una solución de (\ref{Duhamel}). Finalizando así la existencia.
       \item \textbf{Unicidad.}
       \item \textbf{Dependencia Continua.}
       \textcolor{blue}{luego hago esto, toca monear}








   \end{itemize}   
\end{proof}


