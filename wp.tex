%!TEX root = main.tex
Consideremos el problema de valor inicial asociado a la \textit{ecuación con no linealidad modificada de Zakharov-Kusnetsov-Burgers}
\begin{equation}\label{ZKB}
    \begin{cases}
    u_t+\partial_{x_1}\Delta u-\Delta u+u^3=0, & (x,t)\in(-\pi,\pi)^n\times(0,\infty),\\
    u(x,0)=u_0(x), & x\in[-\pi,\pi]^n.
\end{cases}
\end{equation}
Procedamos de manera formal en búsqueda de un candidato a solución, tomando la transformada de Fourier respecto a la variable espacial
\begin{align*}
   (u_t+\partial_{x_1}\Delta u-\Delta u+u^3)^{\wedge}(k)&= \widehat{u}_t(k)+\widehat{\partial_{x_1}\Delta u}(k)-\widehat{\Delta u}(k)+\widehat{u^3}(k)\\
   &=\partial_t\widehat{u}(k)+ik_1\widehat{\Delta u}(k)-\widehat{\Delta u}(k)+\widehat{u^3}(k)\\
   &=\partial_t\widehat{u}(k)+(ik_1-1)\sum_{i=1}^n\widehat{\partial^2_{x_i}u}(k)+\widehat{u^3}(k)\\
   &=\partial_t\widehat{u}(k)+(ik_1-1)\sum_{i=1}^ni^2k_i^2\widehat{u}(k)+\widehat{u^3}(k)\\
   &=\partial_t\widehat{u}(k)+(1-ik_1)|k|^2\widehat{u}(k)+\widehat{u^3}(k).\\
\end{align*}
Así junto al hecho de que $\widehat{u}(k,0)=\widehat{u}_0(k)$ para todo $k\in\Z$ tenemos una ecuación diferencial ordinaria asociada a un problema de valor inicial respecto a la variable temporal 
$$\begin{cases}
    \dfrac{d}{dt}\widehat{u}(k)+(|k|^2-ik_1|k|^2)\widehat{u}(k)=-\widehat{u^3}(k), & k\in\Z^n,t>0\\
    \widehat{u}(k,0)=\widehat{u}_0(k), &k\in\Z.
\end{cases}$$
Luego usando el factor integrante $e^{(|k|^2-ik_1|k|^2)t}$, e integrando a ambos lados de 0 a $t$ tenemos que
$$e^{(|k|^2-ik_1|k|^2)t}\widehat{u}(k,t)-\widehat{u}_0(k)=-\int_0^te^{(|k|^2-ik_1|k|^2)t^\prime}\widehat{u^3}(k,t^\prime)\,dt^\prime.$$
Así despejando $\widehat{u}(k,t)$ llegamos a que
$$\widehat{u}(k,t)=e^{(ik_1|k|^2-|k^2|)t}\widehat{u}_0(k)-\int_0^te^{(ik_1|k|^2-|k^2|)(t-t^\prime)}\widehat{u^3}(k,t^\prime)\,dt^\prime,$$
tomando la transformada inversa de Fourier tenemos que
\begin{align*}
    u(x,t)&=(e^{(ik_1|k|^2-|k^2|)t}\widehat{u}_0(k))^{\vee}-\int_0^t(e^{(ik_1|k|^2-|k^2|)(t-t^\prime)}\widehat{u^3}(k,t^\prime))^{\vee}\,dt^\prime\\
    &=\sum_{k\in\Z^n}e^{(ik_1|k|^2-|k^2|)t+ik\cdot x}\widehat{u}_0(k)-\int_0^t\sum_{k\in\Z^n}e^{(ik_1|k|^2-|k^2|)(t-t^\prime)+ik\cdot x}\widehat{u^3}(k,t^\prime)\,dt^\prime.
\end{align*}
Teniendo en cuenta esto, es en cierta medida natural definir la familia de operadores $\{U(t)\}_{t\geq 0}$ tal que:
$$U(t)f(x)=\sum_{k\in\Z^n}e^{(ik_1|k|^2-|k^2|)t+ik\cdot x}\widehat{f}(k).$$
Para $f$ lo suficientemente regular. De esta manera, nuestro candidato a solución es la formula de Duhamel dada por
\begin{equation}\label{Duhamel}
u(x,t)=U(t)u_0(x)-\int_0^tU(t-t^\prime)u^3(x,t^\prime)\,dt^\prime.
\end{equation}
\begin{definition}
    Dado $s\geq 0.$ Diremos que el problema (\ref{ZKB}) esta localmente bien planteado en $H^s(\T^n)$ si:
    \begin{itemize}
        \item \textbf{(Existencia y Unicidad).} Dado $u_0\in H_s(\T^n),$ existe $T>0$ y una única solución de la formula de Duhamel (\ref{Duhamel}) con dato inicial $u_0$ en el espacio $C([0,T];H^s(\T^n)).$
        \item \textbf{(Dependencia Continua).} Dado $u_0\in H^s(\T^n)$ existe una vecindad $V$ de $u_0$ en $H^s(\T^n)$ y $T>0$ tal que la aplicación dato inicial solución 
        $$v_0\in H^s(\T^n)\mapsto v\in C([0,T];H^s(\T^n)). $$ 
        Es continua.
    \end{itemize}
\end{definition}