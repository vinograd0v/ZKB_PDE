\documentclass[12pt]{article}
\usepackage{float}
\usepackage{xcolor}
\usepackage{newpxtext,euler}
\usepackage[OT1]{fontenc}
\usepackage[spanish]{babel}
\usepackage{amsfonts,amsmath,amssymb,amsthm}
\usepackage{geometry}

\newcommand\N{\ensuremath{\mathbb{N}}}
\newcommand\R{\ensuremath{\mathbb{R}}}
\newcommand\Z{\ensuremath{\mathbb{Z}}}
\renewcommand\O{\ensuremath{\emptyset}}
\newcommand\Q{\ensuremath{\mathbb{Q}}}
\newcommand\C{\ensuremath{\mathbb{C}}}
\renewcommand{\epsilon}{\varepsilon}

\setlength{\parindent}{0pt}


% Tcolorboxes
\makeatother
\usepackage{thmtools}
\usepackage[framemethod=TikZ]{mdframed}
\mdfsetup{skipabove=1em,skipbelow=1em}

\theoremstyle{definition}
\declaretheoremstyle[
    headfont=\bfseries\sffamily\color{black!70!black}, bodyfont=\normalfont,
    mdframed={
        linewidth=1pt,
        rightline=true, topline=true, bottomline=true,
        linecolor=black, backgroundcolor=black!1!white,
    }
]{thmbox}
\declaretheoremstyle[
    headfont=\bfseries\sffamily\color{black!70!black}, bodyfont=\normalfont,
    mdframed={
        linewidth=1pt,
        leftline=false,rightline=false, topline=false, bottomline=false,
        linecolor=black, backgroundcolor=black!1!white,
    }
]{standar}


\declaretheoremstyle[
    headfont=\bfseries\sffamily\color{black!70!black}, bodyfont=\normalfont,
    numbered=no,
    mdframed={
        linewidth=0pt,
        rightline=false, topline=false, bottomline=false,
        linecolor=black, backgroundcolor=black!0!white,
    },
    qed=\qedsymbol
]{thmproofbox}

\declaretheorem[numberwithin=section,style=thmbox, name=Definición]{definition}
\declaretheorem[sibling=definition,style=standar, numbered=no, name=Ejemplo]{eg}
\declaretheorem[sibling=definition,style=thmbox, name=Proposición]{prop}
\declaretheorem[sibling=definition,style=thmbox, name=Teorema, numbered=yes]{theorem}
\declaretheorem[sibling=definition,style=thmbox, name=Lema]{lemma}
\declaretheorem[sibling=definition,style=thmbox, name=Corolario]{corollary}

\declaretheorem[style=thmproofbox, name=Demostración]{replacementproof}
\renewenvironment{proof}[1][\proofname]{\vspace{-10pt}\begin{replacementproof}}{\end{replacementproof}}

\declaretheorem[style=standar, numbered=no, name=Nota]{note}


\newcommand{\bb}[1]{\mathbb{#1}}

\usepackage[most]{tcolorbox}
\tcbuselibrary{most}

\tcbset {
  base/.style={
    arc=7mm, 
    bottomtitle=0.5mm,
    boxrule=0mm,
    colbacktitle=black!90!white, 
    coltitle=white, 
    fonttitle=\bfseries, 
    left=2.5mm,
    leftrule=1mm,
    right=3.5mm,
    title={#1},
    toptitle=0.75mm, 
  }
}


\newtcolorbox{subbox}[1]{
  colframe=black!93!white,
  base={#1}
}
\usepackage{lipsum}
 \geometry{
 a4paper,
 total={170mm,260mm},
 left=20mm,
 top=15mm,
 }
 
\title{\vspace{-2cm}\par\noindent\rule{16cm}{1pt}\large
\\\bfseries Trabajo final de edp, luego vemos un buen título\\
xddddddddddddddddddddddddddddddd
\vspace{-0.34cm}\par\noindent\hspace{0.15cm}\rule{16cm}{1pt}
\vspace{-0.6cm}
}
\author{\small \bfseries María Alejandra Rodríguez Ríos$.^1$\quad \quad\small Mateo Andrés Manosalva Amaris$.^{2}$\\ \small \texttt{mrodriguezri@unal.edu.co} \quad \quad \quad \quad \quad \quad \quad \quad \texttt{mmanosalva@unal.edu.co}\\ \small \bfseries Edgar Santiago Ochoa Quiroga\\
\small \texttt{eochoaq@unal.edu.co}
}

\usepackage{titling}
\predate{\hspace{6.24cm}\small}
\postdate{}

\begin{document}

\maketitle
\begin{abstract}
\lipsum[1]
\end{abstract}

\section{Preliminares}

\begin{definition}
 Sea $\{e_j:  1\leq j\leq n\}$ la base canónica de $\R^n$, una función $f: \mathbb{R}^n \rightarrow \mathbb{C}$ se dice periódica de periodo $L \neq 0$, si

$$
f(x+L e_j)=f(x), \text { para todo } x \in \mathbb{R}^n \text{ y } 1\leq j\leq n, 
$$


\end{definition}

Algunas observaciones:
\begin{itemize}
    \item  Tenemos que dado $L\in \R-\{0\}$, para todo $m \in \mathbb{Z}, m L$ también es un periodo de $f$ en el sentido que $f(x+m L e_j)=f(x)$, para cualquier $x \in \mathbb{R}^n $ y $1\leq j\leq n$. En particular, $-L$ también es un periodo para la función $f$. Por esta razón se puede asumir que $L>0$.
    
    \item Si $f$ es constante, entonces $f$ es periódica de cualquier periodo. Si $f$ es periódica no constante, existe un menor periodo $L>0$, el cual se conoce como periodo fundamental.
\end{itemize}

\begin{note}
    Dado $L>0$ el conjunto $C_{\text {per }}([-L, L]^n)$ consiste de todas las funciones $f: \mathbb{R}^n \rightarrow \mathbb{C}$ continuas con periodo $2L$. Similarmente, dado $k \in \mathbb{Z}^{+}, C_{p e r}^k([-L, L]^n)$ consiste de todas las funciones $f: \mathbb{R}^n \rightarrow \mathbb{C}$ periódicas con periodo $2 L$ de clase $C^k$.\\

De manera equivalente, los espacios anteriores se pueden identificar como sigue:

$$
C_{\text {per }}([-L, L]^n):=\{f \in C([-L, L]^n): f(Le_j)=f(-Le_j), 1\leq j\leq n\}.
$$


Cuando $k \in \mathbb{Z}^{+}$, dado un multiíndice $\beta=(\beta_1,\ldots,\beta_n)$,

\begin{align*}
    C_{\text {per }}^k([-L, L]^n):&=\left\{f \in C^k([-L, L]^n): \partial^{\beta}f(Le_j)=\partial^{\beta}f(-Le_j), \text{ para todo } \beta \text{ tal que } |\beta|\leq k\right. \\
    &\hspace{0.8cm} \text{y para todo } 1\leq j\leq n\}
\end{align*}

Diremos que $f \in C_{\text {per }}^{\infty}([-L, L]^n)$, si $f \in C_{\text {per }}^k([-L, L]^n)$ para cualquier $k \in \mathbb{Z}^{+}$. Por consistencia en la notación, definimos $C_{p e r}^0([-L, L]^n)=C_{p e r}([-L, L]^n)$.\\

De la definición de los espacios $C_{p e r}^k([-L, L]^n)$ notamos que

$$2
C_{p e r}([-L, L]^n) \supseteq C_{p e r}^1([-L, L]^n) \supseteq \cdots \supseteq C_{p e r}^k([-L, L]^n).
$$

\end{note}

Dado $L>0$, si $f: \mathbb{R}^n \rightarrow \mathbb{C}$ es una función periódica de periodo $2 L$, entonces

$$
\widetilde{f}(x):=f\left(\frac{L}{\pi} x\right)
$$

define una función periódica de periodo $2 \pi$. En particular, obtenemos que los espacios $C_{p e r}^k([-L, L]^n)$ y $C_{p e r}^k([-\pi, \pi]^n)$ son isomorfos como espacios vectoriales. De igual manera, la teoría de la transformada de Fourier se puede reescalar para pasar de la región $[-L, L]^n a[-\pi, \pi]^n$. Por estas razones $y$ sin pérdida de generalidad, en lo que sigue trabajaremos con funciones periódicas de periodo $2 \pi$.\\

Otra notación frecuente para funciones periódicas y transformada de Fourier es considerar funciones sobre el toro $\mathbb{T}^n$. El toro $\mathbb{T}^n$ es el intervalo $[0,2 \pi]^n$ donde los lados opuestos se identifican. De manera más precisa, el toro es el conjunto de clases de equivalencia en $\mathbb{R}^n$ dada por la relación $x \sim y$ si $y$ solo si $x-y \in 2 \pi \mathbb{Z}^n$, es decir $\mathbb{T}^n=\mathbb{R}^n/ 2 \pi \mathbb{Z}^n$. Adicionalmente el toro es un grupo aditivo por lo que será útil hacerle una traslación a $[-\pi,\pi]^n$\\

Por las propiedades anteriores, funciones $f: \mathbb{T}^n \rightarrow \mathbb{C}$ se pueden identificar como funciones periódicas $f: \mathbb{R}^n \rightarrow \mathbb{C}$ con periodo $2 \pi$. De esta manera, $C_{\text {per }}^k([-\pi, \pi]^n)$ es isomorfo a $C^k(\mathbb{T}^n)$ por lo que por comodidad trabajaremos sobre este espacio. \cite{ogrianoc}\\

\begin{note}
    Dado que  podemos identificar $\mathbb{T}^n$ como $[-\pi,\pi]^n$, vemos que la integración de funciones sobre el toro resulta de restringir la medida de Lebesgue en $[\pi,\pi]^n$ y por la periodicidad de las funciones en $\mathbb{T}^n$ tenemos que:

    $$\int_{\mathbb{T}^n} f(x) d x=\int_{[-\pi,\pi]^n} f(x) d x=\int_{[0,2\pi]^n} f(x) d x=\int_{\left[a_1, 2\pi+a_1\right] \times \cdots \times\left[a_n, 2\pi+a_n\right]} f(x) d x$$


    para cualesquiera $a_1,\ldots,a_n \in \R$, en efecto:

    $$\begin{aligned}
\int_{[\pi, \pi]^n} f(x) d x & =\int_{[-\pi, 0]^n} f(x) d x+\int_{[0, \pi]^n} f(x) d x \\
& =\int_{[\pi, 2 \pi]^n} f(y-2 \pi \cdot \mathbf{1}) d y+\int_{[0, \pi]^n} f(x) d x \\
& =\int_{[\pi, 2 \pi]^n} f(x) d x+\int_{[0, \pi]^n} f(x) d x\\
&=\int_{[-0,2\pi]^n} f(x) d x
\end{aligned}$$
    

\end{note}

    Finalmente por la periodicidad, la integración por partes no nos deja términos de borde:

    \begin{align*}
        \int_{\mathbb{T}^n} \partial_j f(x) g(x) d x&=f(x)g(x)\big|_{\partial \mathbb{T}^n}-\int_{\mathbb{T}^n} \partial_j g(x) f(x) d x\\
        &=-\int_{\mathbb{T}^n} \partial_j g(x) f(x) d x
    \end{align*}

ya que $f(-\pi e_j)g(-\pi e_j)=f(\pi e_j)g(\pi e_j)$ para todo $1\leq j\leq n$\\

\textcolor{red}{Esto está bien?}
\textcolor{blue}{Parece que si}

\section{Resultados de Buen Planteamiento}

Consideremos el problema de valor inicial asociado a la \textit{ecuacion con no linealidad modificada de Zakharov-Kustnesov-Burgers}
$$\begin{cases}
    u_t+\partial_{x_1}\Delta u-\Delta u+u^3=0, & (x,t)\in(-\pi,\pi)^n\times(0,\infty),\\
    u(x,0)=u_0(x), & x\in[-\pi,\pi]^n.
\end{cases}$$
\textcolor{blue}{Luego escribo bien, de momento pura cuenta}
\begin{align*}
   (u_t+\partial_{x_1}\Delta u-\Delta u+u^3)\hat{\phantom{i}}(k)&= \widehat{u}_t(k)+\widehat{\partial_{x_1}\Delta u}(k)-\widehat{\Delta u}(k)+\widehat{u^3}(k)\\
   &=\partial_t\widehat{u}(k)+ik_1\widehat{\Delta u}(k)-\widehat{\Delta u}(k)+\widehat{u^3}(k)\\
   &=\partial_t\widehat{u}(k)+(ik_1-1)\sum_{i=1}^n\widehat{\partial^2_{x_i}u}(k)+\widehat{u^3}(k)\\
   &=\partial_t\widehat{u}(k)+(ik_1-1)\sum_{i=1}^ni^2k_i^2\widehat{u}(k)+\widehat{u^3}(k)\\
   &=\partial_t\widehat{u}(k)+(1-ik_1)|k|^2\widehat{u}(k)+\widehat{u^3}(k).\\
\end{align*}
Asi junto al hecho de que $\widehat{u}(k,0)=\widehat{u}_0(k)$ para todo $k\in\Z$ tenemos el siguiente problema de valor inicial (EDO) respecto a $t$
$$\begin{cases}
    \dfrac{d}{dt}\widehat{u}(k)+(|k|^2-ik_1|k|^2)\widehat{u}(k)=-\widehat{u^3}(k), & k\in\Z^n,t>0,\\
    \widehat{u}(k,0)=\widehat{u}_0(k), &k\in\Z.
\end{cases}$$
Luego usando el factor integrante $e^{(|k|^2-ik_1|k|^2)t}$, e integrando a ambos lados de 0 a $t$ tenemos que
$$e^{(|k|^2-ik_1|k|^2)t}\widehat{u}(k,t)-\widehat{u}_0(k)=-\int_0^te^{(|k|^2-ik_1|k|^2)t^\prime}\widehat{u^3}(k,t^\prime)\,dt^\prime.$$
Asi despejando $\widehat{u}(k,t)$ llegamos a que
$$\widehat{u}(k,t)=e^{(ik_1|k|^2-|k^2|)t}\widehat{u}_0(k)-\int_0^te^{(ik_1|k|^2-|k^2|)(t-t^\prime)}\widehat{u^3}(k,t^\prime)\,dt^\prime,$$
Tomando la transformada inversa de Fourier tenemos que


\newpage
\bibliographystyle{unsrt}

\bibliography{references}
\nocite{*}

\end{document}


