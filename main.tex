\documentclass[12pt]{article}
\usepackage{float}
\usepackage{xcolor}
\usepackage{newpxtext,euler}
\usepackage[OT1]{fontenc}
\usepackage[spanish]{babel}
\usepackage{amsfonts,amsmath,amssymb,amsthm}
\usepackage{geometry}
\usepackage{bm} % for \bm
\usepackage{fixmath}

\newcommand\N{\ensuremath{\mathbb{N}}}
\newcommand\R{\ensuremath{\mathbb{R}}}
\newcommand\Z{\ensuremath{\mathbb{Z}}}
\renewcommand\O{\ensuremath{\emptyset}}
\newcommand\Q{\ensuremath{\mathbb{Q}}}
\newcommand\C{\ensuremath{\mathbb{C}}}
\newcommand\T{\mathbb{T}}
\renewcommand{\epsilon}{\varepsilon}
\renewcommand{\hat}{\widehat}
\newcommand\jk{\langle k\rangle}


\setlength{\parindent}{0pt}


% Tcolorboxes
\makeatother
\usepackage{thmtools}
\usepackage[framemethod=TikZ]{mdframed}
\mdfsetup{skipabove=1em,skipbelow=1em}

\theoremstyle{definition}
\declaretheoremstyle[
    headfont=\bfseries\sffamily\color{black!70!black}, bodyfont=\normalfont,
    mdframed={
        linewidth=1pt,
        rightline=true, topline=true, bottomline=true,
        linecolor=black, backgroundcolor=black!1!white,
    }
]{thmbox}
\declaretheoremstyle[
    headfont=\bfseries\sffamily\color{black!70!black}, bodyfont=\normalfont,
    mdframed={
        linewidth=1pt,
        leftline=false,rightline=false, topline=false, bottomline=false,
        linecolor=black, backgroundcolor=black!1!white,
    }
]{standar}


\declaretheoremstyle[
    headfont=\bfseries\sffamily\color{black!70!black}, bodyfont=\normalfont,
    numbered=no,
    mdframed={
        linewidth=0pt,
        rightline=false, topline=false, bottomline=false,
        linecolor=black, backgroundcolor=black!0!white,
    },
    qed=\qedsymbol
]{thmproofbox}

\declaretheorem[numberwithin=section,style=thmbox, name=Definición]{definition}
\declaretheorem[sibling=definition,style=standar, numbered=no, name=Ejemplo]{eg}
\declaretheorem[sibling=definition,style=thmbox, name=Proposición]{prop}
\declaretheorem[sibling=definition,style=thmbox, name=Teorema, numbered=yes]{theorem}
\declaretheorem[sibling=definition,style=thmbox, name=Lema]{lemma}
\declaretheorem[sibling=definition,style=thmbox, name=Corolario]{corollary}

\declaretheorem[style=thmproofbox, name=Demostración]{replacementproof}
\renewenvironment{proof}[1][\proofname]{\vspace{-10pt}\begin{replacementproof}}{\end{replacementproof}}

\declaretheorem[style=standar, numbered=no, name=Nota]{note}


\newcommand{\bb}[1]{\mathbb{#1}}

\usepackage[most]{tcolorbox}
\tcbuselibrary{most}

\tcbset {
  base/.style={
    arc=7mm, 
    bottomtitle=0.5mm,
    boxrule=0mm,
    colbacktitle=black!90!white, 
    coltitle=white, 
    fonttitle=\bfseries, 
    left=2.5mm,
    leftrule=1mm,
    right=3.5mm,
    title={#1},
    toptitle=0.75mm, 
  }
}


\newtcolorbox{subbox}[1]{
  colframe=black!93!white,
  base={#1}
}
\usepackage{lipsum}
 \geometry{
 a4paper,
 total={170mm,260mm},
 left=20mm,
 top=15mm,
 }
 
\title{\vspace{-2cm}\par\noindent\rule{16cm}{1pt}\large
\\\bfseries Sobre la buena colocación del problema de Cauchy \\
asociado a la ecuación con no linealidad modificada de Zakharov-Kusnetsov-Burgers en espacios de Sobolev $\mathbf{H^s}\pmb{(}\pmb{\mathbb{T}}\mathbf{^n}\pmb{)}.$
\vspace{-0.34cm}\par\noindent\hspace{0.15cm}\rule{16cm}{1pt}
\vspace{-0.6cm}
}
\author{\small \bfseries María Alejandra Rodríguez Ríos$.^1$\quad \quad\small Mateo Andrés Manosalva Amaris$.^{2}$\\ \small \quad \texttt{mrodriguezri@unal.edu.co} \quad \quad \quad \quad \quad \quad \texttt{mmanosalva@unal.edu.co}\quad\quad \quad\\ \small \bfseries Edgar Santiago Ochoa Quiroga$^{3}$\\
\small \texttt{eochoaq@unal.edu.co}
}

\usepackage{titling}
\predate{\hspace{6.24cm}\small}
\postdate{}

\begin{document}

\maketitle
\begin{abstract}
Este trabajo se centra en el estudio de la ecuación no lineal modificada de Zakharov-Kusnetsov-Burgers (ZKB) en $\T^n$, el cual es un dominio periódico. Utilizando resultados de series de Fourier y espacios de Sobolev, se realiza la buena colocación local del problema en dichos espacios. El resultado principal demuestra que, para $n \geq 1$ y $s > \frac{n}{2}$, existe una única solución local en el tiempo que depende del dato inicial en $H^s(\mathbb{T}^n)$, se realiza la demostración en base a la fórmula integral de Duhamel, lo que permite un análisis formal de la dependencia continua de las soluciones.
\end{abstract}

\section{Introducción}
En este trabajo vamos a realizar un estudio de la ecuación no lineal modificada de Zakharov-Kusnetsov-Burgers,

\begin{equation}\label{ZKB1}
    \begin{cases}
    u_t+\partial_{x_1}\Delta u-\Delta u+u^3=0, & (x,t)\in(-\pi,\pi)^n\times(0,\infty),\\
    u(x,0)=u_0(x), & x\in[-\pi,\pi]^n.
\end{cases}
\end{equation}
donde $u_0$ es un dato inicial en algún espacio de Sobolev periódico, que durante este trabajo denotaremos como $H^s(\mathbb{T}^n)$, $s \geq 0$. \\

Utilizando la transformada de Fourier en la ecuación anterior, de manera formal  se llegará a la fórmula integral, la cual para (\ref{ZKB1}) es 

\begin{equation*}\label{Duhamel}
u(x,t)=U(t)u_0(x)-\int_0^tU(t-t^\prime)u^3(x,t^\prime)\,dt^\prime.
\end{equation*}
      
Aunque nos enfocaremos en el estudio de la ecuación no lineal modificada de Zakharov-Kuznetsov-Burgers, es pertinente mencionar brevemente las ecuaciones de Zakharov-Kuznetsov y Zakharov-Kuznetsov-Burgers, que son ampliamente utilizadas para describir ondas no lineales en medios fuertemente magnetizados. La ecuación de Zakharov-Kuznetsov se ha empleado para investigar ondas ionoacústicas en plasmas. Esta ecuación permite modelar la evolución isotrópica que describe la propagación de ondas débiles en dos o más dimensiones. Por otro lado, la ecuación de Zakharov-Kuznetsov-Burgers, la cual es un análogo bidimensional de la conocida ecuación de Korteweg-de Vries-Burgers, amplía el modelo dado por la ecuación de Zakharov-Kuznetsov  al incorporar efectos de disipación y dispersión, haciéndola aplicable a situaciones con amortiguamiento, como en plasmas con viscosidad o fricción. Ambas ecuaciones son de gran relevancia en campos como la física de plasmas y la mecánica de fluidos, así como en otras áreas que involucran la propagación de ondas no lineales. Su estructura la hace muy útil en el ámbito de las ecuaciones dispersivas, permitiendo estudiar la dinámica de las soluciones, la estabilidad y los comportamientos asintóticos de ondas viajeras en sistemas multidimensionales. Todo lo anterior, se puede evidenciar en \cite{texto1},\cite{texto2},\cite{texto3},\cite{texto4},\cite{texto5}.\\

Este trabajo está organizado de la siguiente manera: en primer lugar, se presenta una sección de preliminares donde se exponen las definiciones y resultados teóricos necesarios para su desarrollo. En esta sección se incluyen resultados sobre espacios periódicos, la transformada de Fourier en $ \T^n$  y los espacios de Sobolev. En la segunda sección, se presentan los resultados sobre el buen planteamiento de la ecuación de Zakharov-Kuznetsov-Burgers no lineal modificada, donde demostramos la existencia y unicidad de la solución a partir de la fórmula integral, además, se aborda la dependencia continua de dicha solución todo lo anterior con respecto al dato inicial $ u_0$ en el espacio $C([0, T]; H^s(\T^n))$.







\section{Preliminares}

\begin{definition}
 Sea $\{e_j:  1\leq j\leq n\}$ la base canónica de $\R^n$, una función $f: \mathbb{R}^n \rightarrow \mathbb{C}$ se dice periódica de periodo $L \neq 0$, si

$$
f(x+L e_j)=f(x), \text { para todo } x \in \mathbb{R}^n \text{ y } 1\leq j\leq n, 
$$


\end{definition}

Algunas observaciones:
\begin{itemize}
    \item  Tenemos que dado $L\in \R-\{0\}$, para todo $m \in \mathbb{Z}, m L$ también es un periodo de $f$, es  decir, $f(x+m L e_j)=f(x)$, para todo $x \in \mathbb{R}^n $ y $1\leq j\leq n$. En particular, $-L$ también es un periodo para la función $f$. Por esto se puede asumir en particular que $L>0$.
    
    \item Si $f$ es periódica no constante, existe un periodo mínimo $L>0$, el cual se conoce como periodo fundamental.
\end{itemize}

\begin{note}
    Dado $L>0$ el conjunto $C_{\text {per }}([-L, L]^n)$ consiste en las funciones $f: \mathbb{R}^n \rightarrow \mathbb{C}$ continuas con periodo $2L$. Análogamente, dado $k \in \mathbb{Z}^{+}, C_{p e r}^k([-L, L]^n)$ consiste de todas las funciones $f: \mathbb{R}^n \rightarrow \mathbb{C}$ periódicas con periodo $2 L$ de clase $C^k$.\\

Una manera equivalente de identificar los espacios anteriores es la siguiente:
$$
C_{\text {per }}([-L, L]^n):=\{f \in C([-L, L]^n): f(Le_j)=f(-Le_j), 1\leq j\leq n\}.
$$


Cuando $k \in \mathbb{Z}^{+}$, dado un multiíndice $\beta=(\beta_1,\ldots,\beta_n)$,

\begin{align*}
    C_{\text {per }}^k([-L, L]^n):&=\left\{f \in C^k([-L, L]^n): \partial^{\beta}f(Le_j)=\partial^{\beta}f(-Le_j), \text{ para todo } \beta \text{ tal que } |\beta|\leq k\right. \\
    &\hspace{0.8cm} \text{y para todo } 1\leq j\leq n\}.
\end{align*}

Diremos que $f \in C_{\text {per }}^{\infty}([-L, L]^n)$, si $f \in C_{\text {per }}^k([-L, L]^n)$ para cualquier $k \in \mathbb{Z}^{+}$.
\end{note}


Note que:

$$
C_{p e r}([-L, L]^n) \supseteq C_{p e r}^1([-L, L]^n) \supseteq \cdots \supseteq C_{p e r}^k([-L, L]^n),
$$

es decir, el conjunto es más pequeño cuando pedimos más regularidad a las funciones.

Dado $L>0$, si $f: \mathbb{R}^n \rightarrow \mathbb{C}$ es una función periódica de periodo $2 L$, entonces el mapeo

$$
\widetilde{f}(x):=f\left(\frac{L}{\pi} x\right),
$$

define una función periódica de periodo $2 \pi$. Esto es, que los espacios $C_{p e r}^k([-L, L]^n)$ y $C_{p e r}^k([-\pi, \pi]^n)$ son isomorfos como espacios vectoriales. Por estas razones y sin pérdida de generalidad, trabajaremos con funciones de periodo $2\pi$.\\

Otra forma frecuente para trabajar sobre este espacio de funciones periódicas es considerar funciones sobre  el toro $\mathbb{T}^n$. El toro $\mathbb{T}^n$ es el intervalo $[0,2 \pi]^n$ donde los lados opuestos se identifican.\\

De manera más rigurosa, el toro es el conjunto de clases de equivalencia de $\mathbb{R}^n$ dadas por la relación $x \sim y$ si y solo si $x-y \in 2 \pi \mathbb{Z}^n$, esto es $\mathbb{T}^n=\mathbb{R}^n/ 2 \pi \mathbb{Z}^n$. Adicionalmente el toro es un grupo aditivo por lo que será útil hacerle una traslación a $[-\pi,\pi]^n$\\

Por las propiedades del toro que hemos mencionado, funciones $f: \mathbb{T}^n \rightarrow \mathbb{C}$ se pueden identificar como funciones periódicas $f: \mathbb{R}^n \rightarrow \mathbb{C}$ con periodo $2 \pi$. De esta manera, $C_{\text {per }}^k([-\pi, \pi]^n)$ es isomorfo a $C^k(\mathbb{T}^n)$, por comodidad en la notación durante el resto de  este trabajo nos referiremos a este espacio como $C^k(\T^n)$. \cite{ogrianoc}\\

\begin{note}
    Dado que  podemos identificar $\mathbb{T}^n$ como $[-\pi,\pi]^n$, vemos que la integración de funciones sobre el toro resulta de restringir la medida de Lebesgue en $[\pi,\pi]^n$ y por la periodicidad de las funciones en $\mathbb{T}^n$ tenemos que:

    $$\int_{\mathbb{T}^n} f(x) d x=\int_{[-\pi,\pi]^n} f(x) d x=\int_{[0,2\pi]^n} f(x) d x=\int_{C} f(x) d x,$$


    con $C=\left[a_1, 2\pi+a_1\right] \times \cdots \times\left[a_n, 2\pi+a_n\right]$ para cualesquiera $a_1,\ldots,a_n \in \R$, en efecto:

    $$\begin{aligned}
    \int_{[\pi, \pi]^n} f(x) d x & =\int_{[-\pi, 0]^n} f(x) d x+\int_{[0, \pi]^n} f(x) d x \\
     & =\int_{[\pi, 2 \pi]^n} f(y-(2 \pi,\ldots,2\pi)) d y+\int_{[0, \pi]^n} f(x) d x \\
    & =\int_{[\pi, 2 \pi]^n} f(x) d x+\int_{[0, \pi]^n} f(x) d x\\
    &=\int_{[0,2\pi]^n} f(x) d x.
    \end{aligned}$$
\end{note}

Para la última propiedad note que:

$$\int_{[0,2\pi]^ n}f(x)dx=\int_{[0,2\pi]}\int_{[0,2\pi]}\ldots\int_{[0,2\pi]}f(x) dx_1\ldots dx_n,$$

dado que $f(x)$ es integrable entonces vale el teorema de fubini, dicho esto, basta ver que el resultado se tiene para una de las integrales, a saber:

$$\int_{[0,2\pi]}f(x)dx=\int_{[a,2\pi+a]}f(x)dx.$$

Si $a>2\pi$, en efecto:

\begin{align*}
    \int_0^{2\pi}f(x)dx&=\int_0^af(x)dx-\int_{2\pi}^af(x) dx\\
    &=\int_{2\pi}^{2\pi+a}f(y-2\pi)dy-\int_{2\pi}^af(x)dx\\
    &=\int_{2\pi}^{2\pi+a}f(x)dx-\int_{2\pi}^af(x)dx\\
    &=\int_a^{2\pi+a}f(x)dx
,\end{align*}

los casos $a\in [0,2\pi]$ y $a<0$ son análogos.

Finalmente por la periodicidad, la integración por partes no nos deja términos de borde:

    \begin{align*}
        \int_{\mathbb{T}^n} \partial_{x_i} f(x) g(x) d x&=\int_{\T ^{n-1}}\left(\int_{-\pi}^{\pi}\partial_{x_i} f(x) g(x)dx_{i}\right)d\widetilde{x}\\
        &=\int_{\T^{n-1}}\left[fg(x_1,\ldots,\pi,\ldots,x_n)-fg(x_1,\ldots,-\pi,\ldots,x_n)\right]d\widetilde{x}\\
        &\hspace*{0.4cm}-\int_{T^{n-1}}\int_{-\pi}^{\pi}f(x)\partial_{x_i}g(x)dx_id\widetilde{x}\\
        &=\int_{\T^{n-1}}0\,d\widetilde{x}-\int_{\T^n}f(x)\partial_{x_i}g(x)dx\\
        &=-\int_{\T^n}f(x)\partial_{x_i}g(x)dx
    \end{align*}

(\textbf{Espacios} $\mathbf{\ell^p}$ ). Sea $1 \leq p<\infty$. Consideramos el siguiente conjunto de sucesiones  a valor complejo:
$$
\ell^p=\ell^p(\mathbb{Z}^n):=\left\{\alpha=\left(\alpha_k\right)_{k \in \mathbb{Z}^n}: \alpha_k \in \mathbb{C} \text { para todo } k, \sum_{k\in \Z^n}\left|\alpha_k\right|^p<\infty\right\}
$$


Note que $l^p$ es un espacio vectorial con la suma $\left(\alpha_k\right)_{k \in \mathbb{Z}^n}+\left(\beta_k\right)_{k \in \mathbb{Z}^n}=$ $\left(\alpha_k+\beta_k\right)_{k \in \mathbb{Z}^n}$ y la multiplicación por escalar $\lambda \cdot\left(\alpha_k\right)_{k \in \mathbb{Z}^n}=\left(\lambda \alpha_k\right)_{k \in \mathbb{Z}^n}, \lambda \in \mathbb{C}$. Adicionalmente, considere la norma:

$$
\|\alpha\|_p=\left(\sum_{k\in \Z^n}\left|\alpha_k\right|^p\right)^{\frac{1}{p}}
$$


Es posible ver que $\left(\ell^p,\|\cdot\|_p\right)$ es un espacio de Banach. Cuando $p=\infty$, definimos $\ell^{\infty}$ como sigue:

$$
\ell^{\infty}=\ell^{\infty}(\mathbb{Z}^n):=\left\{\alpha=\left(\alpha_k\right)_{k \in \mathbb{Z}^n}: \alpha_k \in \mathbb{C} \text { para todo } k, \sup _{k \in \mathbb{Z}^n}\left|\alpha_k\right|<\infty\right\}
$$

y definimos la norma

$$
\|\alpha\|_{\infty}=\sup _{k \in \mathbb{Z}^n}\left|\alpha_k\right|
$$


También tenemos que $\left(\ell^{\infty},\|\cdot\|_{\infty}\right)$ es un espacio de Banach.
Cuando $p=2$, la norma es inducida por el producto interno

$$
(\alpha \mid \beta)=\sum_{k\in \Z^n} \alpha_k \overline{\beta_k}
$$

donde $\alpha=\left(\alpha_k\right)_{k \in \mathbb{Z}^n}, \beta=\left(\beta_k\right)_{k \in \mathbb{Z}^n} \in \ell^2(\mathbb{Z}^n)$.\\



(\textbf{Norma} $\mathbf{L^p}$ ). Para $1 \leq p<\infty$ y $f \in C(\T^n)$ definimos la norma

$$
\|f\|_p=\left(\int_{\T^n}|f(x)|^p d x\right)^{\frac{1}{p}}
$$


Tenemos que $\left(C(\T^n),\|\cdot\|_p\right)$ es un espacio vectorial normado que no es completo.
Por otro lado, considere la norma

$$
\|f\|_{\infty}=\sup _{x \in \T^n}|f(x)| .
$$


Se puede mostrar también que $\left(C(\T^n),\|\cdot\|_{\infty}\right)$ es un espacio de Banach.
Cuando $p=2$, la norma es inducida por el producto interno

$$
(f \mid g)=\int_{\T^n} f(x) \overline{g(x)} d x
$$


La norma $\|\cdot\|_p$ se le llama la norma $L^p, 1 \leq p \leq \infty$.


\begin{theorem}\label{ortogonalidad}
Si $k \in \mathbb{Z}^n$, sea $\Phi_k(x):=e^{i k \cdot x}$ entonces que para $k, m \in \mathbb{Z}^n$

$$
\begin{array}{ll}
\displaystyle\left(\Phi_k \mid \Phi_m\right)=\int_{\T^n} \Phi_k(x) \overline{\Phi_m(x)}=\begin{cases}
0 \quad &\text{si }m\neq  k,\\
(2\pi)^n &\text{si } m=k
\end{cases}
\end{array}
$$
\end{theorem}

\begin{proof}
En efecto, por el teorema de Tychonoff tenemos que el producto cartesiano de conjuntos compactos es compacto y como $\T^{n}=[-\pi,\pi]\times [-\pi,\pi]\times\ldots\times[-\pi,\pi]$ entonces $\T^{n}$ es compacto, por lo cual, como $\Phi_k(x)$ es una función continua en $[-\pi,\pi]^n$ podemos aplicar el teorema de Fubini como sigue:

\begin{align*}
    \left(\Phi_k \mid \Phi_m\right)&=\int_{[-\pi,\pi]^n}e^{ik \cdot x}e^{-im\cdot x}dx\\
    &=\int_{[-\pi,\pi]}\int_{[-\pi,\pi]}\ldots \int_{[-\pi,\pi]}e^{i(k-m) \cdot x}dx_1\ldots dx_n\\
    &=\int_{[-\pi,\pi]}\int_{[-\pi,\pi]}\ldots \int_{[-\pi,\pi]}e^{i(k_1-m_1) x_1}e^{i(k_2-m_2) x_2}\ldots e^{i(k_n-m_n) x_n}dx_1\ldots dx_n\\
    &=\int_{[-\pi,\pi]}e^{i(k_1-m_1) x_1}dx_1\int_{[-\pi,\pi]}e^{i(k_2-m_2) x_2}dx_2\ldots \int_{[-\pi,\pi]} e^{i(k_n-m_n) x_n}dx_n
.\end{align*}

y como:

$$
\begin{aligned}
\int_{-\pi}^\pi e^{i(k_i-m_i)x_i} d x_i= \begin{cases}0, & \text { si } m_i \neq k_i, \\
2 \pi, & \text { si } m_i=k_i.\end{cases}
\end{aligned}
$$

entonces se concluye que:

$$(\Phi_k|\Phi_m)=\begin{cases}
0, \quad &\text{si } m\neq k,\\
(2\pi)^n &\text{si } m=k.
\end{cases}$$

\end{proof}

Dado el sistema ortogonal $\phi_k(x)=e^{ik\cdot x}$ con $k\in \Z^n$  y $x\in \R^n$, queremos escribir a $f\in C(\mathbb{T}^n)$ como:

$$f(x)=\sum_{k\in \Z^n}c_ke^{ik\cdot x},$$

procedamos formalmente asumiendo que la serie anterior converge uniformemente, así:

\begin{align*}
    \left(f|\Phi_m\right)&=\sum_{k\in \Z^n} c_k\left(\Phi_k|\Phi_m\right)\\
    &=c_m (2\pi)^n
,\end{align*}

donde $c_m$, con $m\in \Z^n$ son los coeficientes de Fourier, esto es:

$$c_m=\dfrac{1}{(2\pi)^n}\left(f|\Phi_m\right)=\dfrac{1}{(2\pi)^n}\int_{\T^n}f(x)e^{-im\cdot x} dx,$$

esto motiva la definición de la transformada y serie de Fourier en $C(\T^n)$.


\begin{definition}
Dada $f \in C(\T^n)$ la secuencia de números complejos $\{\widehat{f}(k)\}_{k \in \mathbb{Z}^n}$ se llama la transformada de Fourier de $f$ y está definida como

$$
\widehat{f}(k)=\frac{1}{(2 \pi)^n} \int_{\T^n} f(x) e^{-i k \cdot x} d x.
$$


Al número complejo $\widehat{f}(k)$ se le llama el coeficiente de Fourier.
La serie (puede ser no convergente)

$$
\sum_{k=\Z^n} \widehat{f}(k) e^{i k \cdot x}.
$$

se llama la serie de Fourier de $f$.
\end{definition}

Se puede estudiar a detalle la convergencia de esta serie, sin embargo esto se sale del propósito de este trabajo. Más adelante presentaremos resultados conocidos de la convergencia de la serie de Fourier que requerimos para obtener nuestro resultado de buena colocación, no adentraremos en los detalles de la demostración de los mismos, esto se puede consultar en \cite{grafakos2008classical} o en \cite{iorio2001fourier}.

\begin{theorem}[Desigualdad de Bessel]\label{bessel}
Si $f \in C (\T^{n})$ entonces:
$$\sum_{k \in \mathbb{Z}^n} | \hat{f}(k) |^2 \leq \frac{1}{(2\pi)^n} \| f \|_{L^2}^2 = \frac{1}{(2\pi)^n} \int_{\T^n} |f(x)|^2 \, dx.$$
\end{theorem}

\begin{proof}
Sabemos que como la norma es mayor o igual a cero, entonces
\begin{align*}
0 &\leq\left\|f(x)-\sum_{\substack{k\in \Z^{n}\\ |k|\leq N}}\widehat{f}(k)e^{ik\cdot x}\right\|^{2}_{L^2}\\
&= || f||_{L^2}^{2}+ \left\| \sum_{\substack{k\in \Z^{n}\\|k|\leq N}} \widehat{f}(k)e^{ik\cdot x}\right\|_{L^2}^{2}- 2 \Re\left(f(x)\left|\sum_{\substack{k \in \Z^{n}\\ |k|\leq N}} \widehat{f}(k) e^{ik\cdot x}\right.\right)\\
&= || f||_{L^2}^{2}+ \sum_{\substack{k,m\in \Z^{n}\\ |k|,|m|\leq N}} \widehat{f}(k)\overline{\widehat{f}(m)}(e^{ik\cdot x}|e^{im\cdot x})- 2 \Re\left(\sum_{\substack{k \in \Z^{n}\\ |k|\leq N}} \overline{\widehat{f}(k)} \left(f(x) |e^{ik\cdot x}\right)\right)\\
&= || f||_{L^2}^{2}+ \sum_{\substack{k\in \Z^{n}\\ |k|\leq N}} (2\pi)^{n}\left|\widehat{f}(k)\right|^{2}-2 \sum_{\substack{k\in \Z^{n}\\ |k|\leq N}}(2\pi)^n\left|\widehat{f}(k)\right|^{2}\\
&= || f||_{L^2}^{2}- \sum_{\substack{k\in \Z^{n}\\ |k|\leq N}}(2\pi)^n\left|\widehat{f}(k)\right|^{2}.
\end{align*}

Así\begin{align*}
0 &\leq|| f||_{[L^2}^{2}-(2\pi)^n \sum_{\substack{k\in \Z^{n}\\ |k|\leq N}}\left|\widehat{f}(k)\right|^{2}.
\end{align*}
Por lo cual 
\begin{align*}
\sum_{\substack{k\in \Z^{n}\\ |k|\leq N}}\left|\widehat{f}(k)\right|^{2} &\leq\dfrac{1}{(2\pi)^n}|| f||_{L^2}^{2}.
\end{align*}
Luego tomando el límite cuando $N\rightarrow \infty$, se sigue que:
\begin{align*}
\sum_{\substack{k\in \Z^{n}}}\left|\widehat{f}(k)\right|^{2} &\leq\dfrac{1}{(2\pi)^n}|| f||_{L^2}^{2}.
\end{align*}
\end{proof}

\begin{theorem}
La transformada de Fourier es lineal.
\end{theorem}

\begin{proof}
Note que:

$$
\begin{aligned}
(f+c g)(k) & =\frac{1}{(2\pi)^n}\int_{\T^n}(f+g)(x) e^{-i k \cdot x} d x \\
& =\frac{1}{(2\pi)^n}\int_{\T^n}(f(x)+c g(x)) e^{-i k \cdot x} d x \\
& =\frac{1}{(2\pi)^n}\int_{\T^n} f(x) e^{-i k \cdot x} d x+\frac{c}{(2\pi)^n} \int_{\T^n} g(x) e^{i k \cdot x} d x \\
& =\hat{f}(k)+c\hat{g}(k)
\end{aligned}
$$
\end{proof}

\begin{theorem}\label{derivadacacorra}
Si $f \in C^l\left(\mathbb{T}^n\right)$ y $\beta=\left(\beta_1, \ldots, \beta_n\right)$ es un multi-índice de orden $l$, es decir, $|\beta|=\beta_1+\cdots+\beta_n=l$, entonces $\widehat{\partial^\beta f}(k)=i^{|\beta|} k^\beta \widehat{f}(k)$, para todo $k \in \mathbb{Z}^n$.
\end{theorem}

\begin{proof}
Procedamos  por inducción sobre el orden del multi-índice, primero note que si $\beta$ es un multi-índice de orden 1, entonces $\beta=(0,\ldots,1,\ldots,0)$, así:

$$\partial^{\beta}f(x)=\partial_{x_j}f(x),$$

luego:

\begin{align*}
    \widehat{\partial^{\beta}f}(k)&=\widehat{\partial_{x_j}f}(k)\\
    &=\dfrac{1}{(2\pi)^n}\int_{\T^n}\partial_{x_j}f(x)e^{-ik\cdot x}\\
    &=-\dfrac{1}{(2\pi)^n}\int_{\T^n}(-ik_j)f(x)e^{-ik\cdot x}\\
    &=\dfrac{i^{|\beta|}k^{\beta}}{(2\pi)^n}\int_{\T^n}f(x)e^{-ik\cdot x}\\
    &=i^{|\beta|}k^{\beta}\hat{f}(k)
.\end{align*}

Ahora supongamos que la igualdad se tiene para todo multi-índice $\alpha$ de orden $n$, note que dado un multi-índice de orden $n+1$, $\beta$, existe un multi-índice $\alpha$ de orden $n$ tal que:

$$\partial^{\beta}f(x)=\partial_{x_j}\partial^{\alpha}f(x),$$

así:

\begin{align*}
    \widehat{\partial^{\beta}f}(k)&=\widehat{\partial_{x_j}\partial^{\alpha}f}(k)\\
    &=ik_j\widehat{\partial^{\alpha}f}(k)\\
    &=ik_ji^{|\alpha|}k^{\alpha}\hat{f}(k)\\
    &=i^{|\beta|}k^{\beta}\hat{f}(k)
.\end{align*}

Por el principio de inducción matemática se sigue la igualdad.

\end{proof}

A continuación presentamos el espacio $L^2(\T^n)$.

\begin{definition}
El espacio $L^2\left(\mathbb{T}^n\right)=L^2\left([-\pi, \pi]^n\right)$ consiste de funciones $f:[-\pi, \pi]^n \rightarrow \mathbb{C}$ medibles según Lebesgue tales que:

$$\|f\|_{L^2}^2=\int_{\T^n}|f(x)|^2 d x<\infty.$$
\end{definition}
Asumimos que,
\begin{theorem}
El espacio $C^{\infty}\left(\mathbb{T}^n\right)$ es denso en $L^2\left(\mathbb{T}^n\right)$.    
\end{theorem}

\begin{note}
Recordemos que $C^{\infty}(\T^n)\subset C^m(\T^ n)\subset  \ldots\subset C(\T^n)$, es claro que las funciones en $C(\T^n)$ son funciones en $L^2(\T^n)$ por la continuidad sobre un compacto, esto nos dice que el conjunto $C^m(\T^n)$ es denso en $L^2(\T^n)$ para todo $m\geq 0$\\

Es posible ver que $L^2(\T)$ es completo y el espacio $C(\T^n)$ no lo es con  la norma $\|\cdot\|_{L^2}$, dicho esto podemos definir a $L^2(\T^n)$ como el espacio de completación de $C(\T^n)$ por la densidad.
\end{note}

\begin{theorem}\label{convergencia}
Si $m>\frac{n}{2}$ con $m$ entero, entonces la serie de Fourier de una función $f \in C^m\left(\mathbb{T}^n\right)$ converge absoluta y uniformemente a $f$, además, se tiene que $\|f\|_{\infty} \leq\|\widehat{f}\|_1$ donde $\|\cdot\|_1$ es la norma de $\ell^1\left(\mathbb{Z}^n\right)$. Más aún, vale la identidad de Parseval

\begin{align}
\|\widehat{f}\|_2^2=\sum_{k \in \mathbb{Z}^n}|\widehat{f}(k)|^2=\frac{1}{(2 \pi)^n} \int_{\T^n}|f(x)|^2 d x=\frac{1}{(2 \pi)^n}\|f\|_{L^2}^2.
\end{align}

De manera equivalente, si $f, g \in C^m\left(\mathbb{T}^n\right)$,

\begin{align}
(\widehat{f} \mid \widehat{g})=\sum_{k \in \mathbb{Z}^n} \widehat{f}(k) \overline{\widehat{g}(k)}=\frac{1}{(2 \pi)^n} \int_{\T^n} f(x) \overline{g(x)} d x=\frac{1}{(2 \pi)^n}(f \mid g).
\end{align}

\end{theorem}

Por lo anterior podemos probar que si $f\in C^{\infty}(\T^{n})$ entonces $\widehat{f}$ es inyectiva.
\begin{theorem}\label{inyectiva}
Sean $f, g \in C^{\infty}(\T^{n})$. Suponga que $\widehat{f}(k) = \widehat{g}(k)$ para todo $k \in \mathbb{Z}^{n}$. Entonces $f = g$.
\end{theorem}
\begin{proof}

 Sea $h = f - g$ luego, por propiedades de las funciones continuas en $\T^{n}$,tenemos que $h \in C^{\infty}(\T^{n})$ y por hipótesis $\hat{h}(k) = 0$ para todo $k \in \mathbb{Z}$, por el Teorema \ref{convergencia}  , se sigue que $\|\hat{h}\|_2= 0=\|h\|_{L^2}$, es decir, $h = 0$.       
\end{proof}
        
\begin{note}
  Sobre la transformada de Fourier en $L^2(\T^n)$ tenemos de la desigualdad de Cauchy-Schwarz:

$$
\begin{aligned}
|\hat{f}(k)| & =\left|\frac{1}{(2 \pi)^n} \int_{\T^n} f(x) e^{-i k \cdot x} d x\right| \\
& =\frac{1}{(2 \pi)^n} |(f|e^{i k \cdot x}) | \\
& \leq \frac{1}{(2 \pi)^n}\|f\|_{L^2}\left\|\Phi_k \right\|_{L^2} \\
& \leq(2 \pi)^{-\frac{n}{2}}\|f\|_{L^2},
\end{aligned}
$$

dado que $\|f\|_{L^2}<\infty$ entonces sabemos que $\widehat{f}=\{\widehat{f}(k)\}\in \ell^{\infty}(\Z^n)$, sin embargo podemos decir aún más sobre la transformada de Fourier en $L^2(\T^n)$  
\end{note}


\begin{theorem}

La transformada de Fourier

$$
\wedge: L^2\left(\mathbb{T}^n\right) \rightarrow \ell^2\left(\mathbb{Z}^n\right)
$$

es un isomorfismo. Además, vale la identidad de Parseval (1) y (2)
\end{theorem}
\begin{proof}
Veamos que si $f \in L^2(\T^n)$, entonces $\widehat{f} \in \ell^2(\mathbb{Z}^{n})$. Sea $\{f_n\}$ una sucesión, de manera que $\{f_n\}\subset C^{m}(\T^{n}) $ tal que $f_n$ converge a $f$ en el sentido de $L^2$. Por el teorema \ref{convergencia}, sabemos que vale la identidad de Parseval para funciones en $C^{m}(\T^{n}) $, lo cual implica

$$
\|f_n - f_m\|^2_{L^2} = (2\pi)^{n} \|\hat{f}_n - \hat{f}_m\|^2_{2},
$$

para todo $n, m \in \mathbb{Z}^{n}$. \\
Luego, como $\{f_n\} \subset C^{m}(\T^{n})$ converge en $L^2$, esta define una secuencia de Cauchy. La identidad de Parseval entonces nos muestra que $\{\widehat{f}_n\}$ es de Cauchy en $\ell^2(\mathbb{Z}^{n})$. Puesto que $\ell^2(\mathbb{Z}^{n})$ es completo, entonces toda sucesión de Cauchy converge. Sea $\{\alpha_k\} \in \ell^2(\mathbb{Z}^{n})$ el límite de $\{\widehat{f}_n\}$.\\

Ahora, veamos que  $\widehat{f} = \{\widehat{f}(k)\} = \{\alpha_k\}$, es decir, $\alpha_k = \widehat{f}(k)$ para todo $k\in \Z^n$.\\

Note que,

$$
|\widehat{f}_n(k) - \alpha_k|^2 \leq \sum_{k\in\Z^n} |\widehat{f}_n(k) - \alpha_k|^2 = \|\hat{f}_n - \{\alpha_k\}\|^2_{2} \to 0 \quad \text{cuando} \quad n \to \infty.
$$

Por otro lado, la desigualdad de Cauchy-Schwarz nos da

$$
|\widehat{f}_n(k) - \widehat{f}(k)| = \dfrac{1}{(2\pi)^{n}} \int_{\T^n} (f_n(x) - f(x)) e^{-ik\cdot x} \, dx \leq (2\pi)^{-\frac{n}{2}} \|f_n - f\|_2 \to 0 \quad\text{cuando}\quad n\to \infty.
$$

Esto es que  $a_k=\widehat{f}(k)$, en efecto $\widehat{f}$ es el límite de $\{ \widehat{f_n} \}$.\\

Con lo anterior es claro que la transformada está llegando $\ell^2(\Z^n)$, pero podemos decir más, note que lo anterior nos dice que $f_n\to f$ en $L^2(\T^n)$ y $\widehat{f_n}\to \hat{f}$ en $\ell^2(\Z^n)$, por la identidad de Parseval tenemos que:

$$\left\|f_n\right\|_{L^2}^2=(2 \pi)^n\left\|\widehat{f_n}\right\|_2^2,$$

tomando $n\to \infty$ se sigue que:

$$\|f\|_{L^2}^2=(2\pi)^n\|\widehat{f}\|_2^2.$$

Esto es que vale la identidad de Parseval en $L^2$. Tenemos que la transformada es lineal, veamos que es inyectiva y continua, en efecto dado $\epsilon>0$, si $\|f-g\|_{L^2}<(2\pi)^{\frac{n}{2}}\epsilon$  por la identidad de Parseval:

\begin{align*}
    \|\widehat{f}-\widehat{g}\|_2=(2\pi)^{\frac{n}{2}}\|f-g\|_{L^2}<\epsilon
.\end{align*}

Análogamente si $\hat{h}(k)=\hat{f}(k)$ para todo $k\in \Z^n$ considere $g=h-f$, entonces $\|\widehat{g}\|_{2}=0=\|g\|_{L^2}$, luego $g=0$. Finalmente debemos ver que $\wedge$ es sobreyectiva.\\

Sea $\left\{\alpha_k\right\} \in \ell^2(\mathbb{Z}^n)$. Dado un entero $n \geq 1$, definimos $f_n(x)=\displaystyle\sum_{\substack{k\in \Z^n\\
|k|\leq n}} \alpha_k \Phi_k$, así $f_n \in C^{\infty}(\T^n) \subset$ $C(\T^n)$. 
Por el Teorema \ref{ortogonalidad} tenemos que:

$$
\left\|f_n-f_m\right\|_{L^2}^2=\left\|\sum_{\substack{k\in \Z^n\\
n+1 \leq|k| \leq m}} \alpha_k \Phi_k\right\|_{L^2}^2=(2 \pi)^n\sum_{\substack{k\in \Z^n\\ n+1 \leq|k| \leq m}}\left|\alpha_k\right|^2
$$


Como $\left\{\alpha_k\right\} \in \ell^2(\mathbb{Z}^n)$, lo anterior muestra que $\left\{f_n\right\}$ es una sucesión de Cauchy en $L^2(\T^n)$. Ahora, como $L^2(\T^n)$ es completo, entonces existe $f \in L^2([-\pi, \pi])$ el límite de $\{f_n\}$ en $L^2$. Basta ver que $\widehat{f}(k)=\alpha_k$ para todo $k \in \mathbb{Z}^n$. Notemos que si $n \geq|k|$, por la construcción de $f_n$, $\widehat{f_n}(k)=\alpha_k$, tomando el límite cuando $n\to \infty$ de $f_n$ obtenemos lo deseado.
\end{proof}

\begin{note}
Sobre la prueba anterior hay un comentario importante, note que nuestro candidato a preimágen de la transformada en la sobreyectividad fue la serie de Fourier de $f$, en este caso es importante resaltar que esta serie es convergente  en el sentido $L^2$.
\end{note}

El teorema anterior implica que existe la inversa de la transformada de Fourier que denotaremos como

$$
\vee: l^2\left(\mathbb{Z}^n\right) \rightarrow L^2\left(\mathbb{T}^n\right)
$$

definida como $\left\{\alpha_k\right\}^{\vee}=\displaystyle\sum_{k \in \mathbb{Z}^n} \alpha_k \Phi_k$, donde el sentido de esta serie es en $L^2(\T^n)$.


\subsection{Espacios de Sobolev}

En esta subsección presentamos los espacios de Sobolev periódicos $H^s(\T^n)$ y algunas propiedades importantes que necesitaremos en la prueba de buena colocación, estos espacios se estudian de manera más detallada  en \cite{ioriojunior2001equacoes}.\\

Es posible demostrar que el espacio $C^1(\T^n)$ no es completo con la norma

$$
\|f\|_{H^1}=\frac{1}{(2 \pi)^{\frac{n}{2}}}\left(\|f\|_{L^2}^2+\left\|f^{\prime}\right\|_{L^2}^2\right)^{\frac{1}{2}}
$$


La norma anterior es interesante para establecer si una función es integrable y diferenciable (en algún sentido) y estudiar si su derivada conserva la misma propiedad de integrabilidad. Esto motiva a definir el espacio \cite{ogrianoc}.

$$
H^1(\T^n)=\overline{\left(C^1(\T^n),\|\cdot\|_{H^1}\right)}.
$$

Dado que la transformada de Fourier en $L^2$ es un isomorfismo y vale la identidad de Parseval en $L^2(\T^n)$ observamos lo siguiente\\


$$
\begin{aligned}
 \|f\|_{H^{1}}&=\frac{1}{(2 \pi)^{\frac{n}{2}}}\left(\sum_{|\alpha|\leq1}\left\|\partial^\alpha f\right\|_{L^2}^2\right)^{\frac{1}{2}} \\
& =\left(\sum_{|\alpha| \leq 1}\| i^{|\alpha|}k^{\alpha} \hat{f}(k) \|_2^2\right)^{\frac{1}{2}} \\
& =\left(\|\widehat{f}(k)\|_2^2+\sum_{j=1}^n\left\|i k_j \hat{f}(k)\right\|_2^2\right)^{\frac{1}{2}} \\
& =\left(\sum_{k \in \Z^n}|\hat{f}(k)|^2+\sum_{k \in \Z^n} \sum_{j=1}^n\left|k_j\right|^2|\hat{f}(k)|^2\right)^{\frac{1}{2}} \\
& =\left(\sum_{k \in \Z^n}|\hat{f}(k)|^2\left(1+|k|^2\right)\right)^{\frac{1}{2}}\\
&=\left(\sum_{k\in\Z^{n}} |\langle k \rangle \hat{f}(k)|^2\right)^{\frac{1}{2}}=\|\langle k\rangle\hat{f}(k)\|_2,
\end{aligned}
$$

donde $\langle k\rangle=(1+|k|^2)^{\frac{1}{2}}$. Esto motiva la siguiente definición.


\begin{definition}
Sea $s \geq 0$. Definimos el espacio de Sobolev periódico de orden $s$ como

$$
H^s=H^s(\T^n)=\left\{f \in L^2(\T^n):\left\{\langle k\rangle^s \widehat{f}(k)\right\} \in \ell^2(\mathbb{Z}^n)\right\},
$$

$\text { para } k=\left(k_1, \ldots, k_n\right) \in \mathbb{Z}^n,\langle k\rangle=\left(1+|k|^2\right)^{\frac{1}{2}} \text { con }|k|^2=k_1^2+\cdots+k_n^2 \text {. }
$
\end{definition}

Le asignamos al espacio $H^s\left(\mathbb{T}^n\right)$ el producto interno

\begin{equation}
(f \mid g)_{H^*}=\sum_{k \in \mathbb{Z}^n}\langle k\rangle^{2 s} \widehat{f}(k) \overline{\widehat{g}(k)}
\end{equation}

para $f, g \in H^s\left(\mathbb{T}^n\right)$, el cual induce la norma

$$
\|f\|_{H^s}^2=\sum_{k \in \mathbb{Z}^n}\langle k\rangle^{2 s}|\widehat{f}(k)|^2
$$

\begin{theorem}
Sea $s \geq 0$. Muestre que $H^s\left(\mathbb{T}^n\right)$ con el producto interno anterior es un espacio de Hilbert.
\end{theorem}

\begin{proof}
Primero veamos que $H^s(\T^n)$ es subespacio vectorial de $L^2(\T^n)$, dados $f,g\in H^s(\T^n)$ y $\lambda\in \C$, en efecto:

\begin{align*}
    \sum_{\substack{k\in \Z^n\\
    |k|\leq N}}|\langle k\rangle^s \hat{f+\lambda g}(k)|^2&=\sum_{\substack{k\in \Z^n\\
    |k|\leq N}}|\langle k\rangle^s( \hat{f}(k)+\lambda \hat{g}(k))|^2\\
    &\leq2\left(\sum_{\substack{k\in \Z^n\\
    |k|\leq N}}|\langle k\rangle^s\hat{f}(k)|^2+|\lambda|^2\sum_{\substack{k\in \Z^n\\
    |k|\leq N}}|\langle k\rangle^s \hat{g}(k)|^2\right)
.\end{align*}

Tomando el límite cuando $N\to \infty$ obtenemos:

$$\|\langle k\rangle^s\hat{f+\lambda g}(k)\|_2^2\leq2 \|\langle k\rangle\hat{f}(k)\|_2^2+2|\lambda|^2\|\langle k\rangle\hat{g}(k)\|_2^2<\infty.$$

Ahora veamos que $H^s(\T^n)$ es completo con la norma inducida por el producto interno (3).\\

Si $(f_j)$ es una sucesión de Cauchy en $H^s(\T^n)$, tenemos que para todo $\varepsilon> 0$, existe $N>0$, tal que si $m,l>N$ entonces
$$\| f_m-f_l\|_{H{s}}< \epsilon $$
Luego, $$\| f_m-f_l\|_{H{s}}=\displaystyle\sum_{k\in \Z^{n}}\langle k\rangle ^{s}|\widehat{f_{m}}(k)-\widehat{f_{l}}(k)|^{2}$$
Por lo cual $\{\langle k \rangle^{s}\widehat{f}_j(k)\}$ es una sucesión de Cauchy en $\ell^{2}(\Z^{n})$, es decir, existe $g\in \ell^{2}(\Z^{n}) $ tal que
$$\langle k \rangle ^{s}\hat{f}(k)\rightarrow g(k)$$
Así, $f=\left(\dfrac{g(k)}{\langle k \rangle^{s}}\right)^{v}$, por tanto $\hat{f}=\left(\dfrac{g(k)}{\langle k \rangle^{s}}\right)$.\\

Para completar la prueba debemos mostrar que $f\in H^{s}$ y que $f_{j}\rightarrow f$ está en $H^{s}$, tenemos que 
$$\|f\|^{s}_{H^{s}}=\displaystyle\sum_{k\in \Z ^{n}} \langle k \rangle^{2s}\left| \dfrac{g(k)}{\langle k\rangle^{s}}\right|^{2}=\displaystyle\sum_{k\in \Z^{n}}|g(k)|^{2}=\|g\|^{2}_{\ell^{2}}<\infty$$
Con lo cual $f\in H^{s}$.\\
Ahora bien, como $f$ es una sucesión de Cauchy entonces para todo $\epsilon>0$, existe $N>0$ de manera que si $j<N$ se tiene que
\begin{align*}
   \|f-f_j\|_{H^{s}}^{2}&=\sum_{k\in \Z}\langle k \rangle^{2s}|\hat{f}(k)-\hat{f_j}(k)|^{2}\\
   &=\sum_{k\in \Z^{n}} \langle k\rangle^{2s}\left| \dfrac{g(k)}{\langle k \rangle^{s}}-  \hat{f_j}(k) \right|^{2} \\
   &= \sum_{k\in \Z^{n}} \left| g(k)- \langle k\rangle^{s} \hat{f_j}(k) \right|^{2} \\
   &< \epsilon^{2}
.\end{align*}
Con lo cual probamos que $H^{s}(\T^{n})$ es completo con la norma inducida por el producto interno.

\end{proof}

\begin{theorem}[Encajamiento de Sobolev]\label{algebra}
Suponga que $s>\frac{n}{2}$, entonces si $f \in H^s\left(\mathbb{T}^n\right)$ se tiene que $f\in C\left(\mathbb{T}^n\right)$ y $\widehat{f} \in l^1\left(\mathbb{Z}^n\right)$. Más precisamente, la aplicación $f \in H^s\left(\mathbb{T}^n\right) \mapsto f \in C\left(\mathbb{T}^n\right)$ es continua $y$ existe una constante universal $C>0$ tal que

$$
\|f\|_{\infty} \leq\|\widehat{f}\|_1 \leq C\|f\|_{H^s}
$$


Más aún, el espacio $H^s\left(\mathbb{T}^n\right)$ define una álgebra de Banach, esto es, existe $C>0$ tal que para todo $f, g \in H^s\left(\mathbb{T}^n\right)$, se tiene que el producto $f g \in H^s\left(\mathbb{T}^n\right)^1 y$

$$
\|f g\|_{H^s} \leq C\|f\|_{H^s}\|g\|_{H^s}
$$
\end{theorem}

\begin{proof}

Sea $f \in H^s(\T^n), s>\frac{n}{2}$. Note que dado $N\geq 1$ por la desigualdad de Cauchy-Schwarz:

$$
\begin{aligned}
\sum_{\substack{k\in\Z^n\\
|k|\leq N}}|\widehat{f}(k)| & =\sum_{\substack{k\in\Z^n\\
|k|\leq N}}\langle k\rangle^{-s}\langle k\rangle^s|\widehat{f}(k)| \\
& \leq\left(\sum_{\substack{k\in\Z^n\\
|k|\leq N}}\langle k\rangle^{-2 s}\right)^{\frac{1}{2}}\left(\sum_{\substack{k\in\Z^n\\
|k|\leq N}}\langle k\rangle^{2 s}|\widehat{f}(k)|^2\right)^{\frac{1}{2}} \\
& \leq\left(\sum_{k\in \Z^n}\langle k\rangle^{-2 s}\right)^{\frac{1}{2}}\left(\sum_{k\in \Z^n}\langle k\rangle^{2 s}|\widehat{f}(k)|^2\right)^{\frac{1}{2}} \\
& =\left(\sum_{k\in \Z^n}\langle k\rangle^{-2 s}\right)^{\frac{1}{2}}\|f\|_{H^s},
\end{aligned}
$$

en efecto:
\begin{align*}
    \sum_{k\in \Z^n}\jk^{-2s}&=\sum_{k\in \Z^n}\frac{1}{(1+|k|^2)^s}\\
    &\leq \sum_{\substack{k\in \Z^n\\
    |k|\neq 0}}\frac{1}{|k|^{2s}}
    ,\end{align*}

es claro que esta serie converge si y solo si la serie:

$$\sum_{k_1,\ldots k_n=1}^{\infty}\frac{1}{(|k_1|^2+|k_2|^2+\ldots+|k_n|^2)^s}.$$

Note que $(a+b)^2=a^2+b^2+o(a^2+b^2)$, el término restante es el orden de o pequeña de $a^2+b^2$, luego basta estudiar la convergencia de la serie:

\begin{align*}
   \sum_{k_1,\ldots k_n=1}^{\infty}\frac{1}{(k_1+k_2+\ldots+k_n)^{2s}}=\sum_{k=1}^{\infty}\frac{a_{k,n}}{k^{2s}}
,\end{align*}

donde $a_{k,n}$ es el número de $n$-tuplas de enteros  positivos cuya suma es $k$. Para cualquier par de enteros positivos $n$ y $k$, el número de $n$-tuplas de enteros positivos cuya suma es $k$ es igual al número subconjuntos de $n - 1$ elementos de un conjunto con $k - 1$ elementos, esto es:

$$a_{k,n}=\binom{k-1}{n-1}= \frac{k!}{(n-1)!(k-n)!}=\frac{k(k-1)\ldots(k-(n-1))}{(n-1)!}\leq \frac{k^{n-1}}{(n-1)!}=O(k^{n-1}).$$ 

Esto es $\displaystyle\sum_{k \in \Z^n}\jk^{-2s}$ converge si y solo si la serie:

$$\sum_{k=1}^{\infty} \frac{k^{n-1}}{k^{2s}}=\sum_{k=1}^{\infty} \frac{1}{k^{2s+1-n}},$$

que es convergente si $s>\frac{n}{2}$ por el criterio de la integral.\\

Así, hemos encontrado una cota para la suma que no depende de $N$ tomando $N \rightarrow \infty$, llegamos a que $\{\widehat{f}(k)\} \in \ell^1(\mathbb{Z}^n)$. Esto nos dice que la serie de Fourier $\displaystyle\sum_{k\in \Z^n} \widehat{f}(k) e^{i k \cdot x}$ converge absoluta y uniformemente, pero como esta también converge a $f$ en $L^2$, por la unicidad del límite en $L^2$, tenemos que $f(x)=\displaystyle\sum_{k\in \Z^n} \widehat{f}(k) e^{i k \cdot x}$ en casi toda parte. Luego, obtenemos que $f$ se identifica con una función continua y por la estimativa previa, ahora veamos que existe una constante $c_s>0$, tal que para todo $k,k_1\in \Z^{n}$ se tiene que 
\begin{align*}
    \langle k \rangle ^{s}\leq c_s \langle k_1\rangle +\langle k-k_1 \rangle^{s}
.\end{align*}
Para esto note que 
\begin{align*}
    \langle k \rangle ^{\frac{s}{2}}&=(1+|k|^{2})^{\frac{s}{2}}\\
    &= (1+|k-k_1+k_1|^{2})^{\frac{s}{2}}\\
    &\leq (1+2|k-k_1|^{2}+2|k_1|^{2})^{\frac{s}{2}}\\
    & <(1+2+2|k-k_1|^{2}+2+2|k_1|^{2})^{\frac{s}{2}}\\
    &\leq (2+|k-k_1|^{2}+|k_1|^{2}+2|k-k_1|^{2}+2|k_1|^{2})^{\frac{s}{2}}\\ 
    &=(6+3|k-k_1|^{2}+3|k_1|^{2})^{\frac{s}{2}}\\
    &= 3^{\frac{s}{2}} (1+|k-k_1|^{2}+1+|k_1|^{2})^{\frac{s}{2}}\\
    &=3^{\frac{s}{2}} (\langle k-k_1\rangle+\langle k_1\rangle)^{\frac{s}{2}}
.\end{align*}
Luego tenemos que 
\begin{align*}
    \langle k \rangle ^{s}&=3^{s} (\langle k-k_1\rangle+\langle k_1\rangle)^{s}
.\end{align*}
Así, por desigualdad de Young
\begin{align*}
    \langle k \rangle ^{s}&\leq C_s (\langle k-k_1\rangle^{s}+\langle k_1\rangle^{s})
.\end{align*}


\end{proof}


%%%%%%%%%%%%%%%%%%%%%%%%%%%%%%%%%%%%%%%%%%%
\section{Resultados de Buen Planteamiento}
%!TEX root = main.tex
Consideremos el problema de valor inicial asociado a la \textit{ecuación con no linealidad modificada de Zakharov-Kusnetsov-Burgers}
\begin{equation}\label{ZKB}
    \begin{cases}
    u_t+\partial_{x_1}\Delta u-\Delta u+u^3=0, & (x,t)\in(-\pi,\pi)^n\times(0,\infty),\\
    u(x,0)=u_0(x), & x\in[-\pi,\pi]^n.
\end{cases}
\end{equation}
Procedamos de manera formal en búsqueda de un candidato a solución, tomando la transformada de Fourier respecto a la variable espacial
\begin{align*}
   (u_t+\partial_{x_1}\Delta u-\Delta u+u^3)^{\wedge}(k)&= \widehat{u}_t(k)+\widehat{\partial_{x_1}\Delta u}(k)-\widehat{\Delta u}(k)+\widehat{u^3}(k)\\
   &=\partial_t\widehat{u}(k)+ik_1\widehat{\Delta u}(k)-\widehat{\Delta u}(k)+\widehat{u^3}(k)\\
   &=\partial_t\widehat{u}(k)+(ik_1-1)\sum_{i=1}^n\widehat{\partial^2_{x_i}u}(k)+\widehat{u^3}(k)\\
   &=\partial_t\widehat{u}(k)+(ik_1-1)\sum_{i=1}^ni^2k_i^2\widehat{u}(k)+\widehat{u^3}(k)\\
   &=\partial_t\widehat{u}(k)+(1-ik_1)|k|^2\widehat{u}(k)+\widehat{u^3}(k).\\
\end{align*}
Así, junto al hecho de que $\widehat{u}(k,0)=\widehat{u}_0(k)$, para todo $k\in\Z^n$ tenemos una ecuación diferencial ordinaria asociada a un problema de valor inicial respecto a la variable temporal 
$$\begin{cases}
    \dfrac{d}{dt}\widehat{u}(k)+(|k|^2-ik_1|k|^2)\widehat{u}(k)=-\widehat{u^3}(k), & k\in\Z^n,t>0.\\
    \widehat{u}(k,0)=\widehat{u}_0(k), &k\in\Z^{n}.
\end{cases}$$
Luego, usando el factor integrante $e^{(|k|^2-ik_1|k|^2)t}$, e integrando a ambos lados de 0 a $t$, tenemos que
$$e^{(|k|^2-ik_1|k|^2)t}\widehat{u}(k,t)-\widehat{u}_0(k)=-\int_0^te^{(|k|^2-ik_1|k|^2)t^\prime}\widehat{u^3}(k,t^\prime)\,dt^\prime.$$
Así, despejando $\widehat{u}(k,t)$ llegamos a que
$$\widehat{u}(k,t)=e^{(ik_1|k|^2-|k|^2)t}\widehat{u}_0(k)-\int_0^te^{(ik_1|k|^2-|k|^2)(t-t^\prime)}\widehat{u^3}(k,t^\prime)\,dt^\prime,$$
tomando la transformada inversa de Fourier 
\begin{align*}
    u(x,t)&=(e^{(ik_1|k|^2-|k|^2)t}\widehat{u}_0(k))^{\vee}-\int_0^t(e^{(ik_1|k|^2-|k|^2)(t-t^\prime)}\widehat{u^3}(k,t^\prime))^{\vee}\,dt^\prime\\
    &=\sum_{k\in\Z^n}e^{(ik_1|k|^2-|k|^2)t+ik\cdot x}\widehat{u}_0(k)-\int_0^t\sum_{k\in\Z^n}e^{(ik_1|k|^2-|k|^2)(t-t^\prime)+ik\cdot x}\widehat{u^3}(k,t^\prime)\,dt^\prime.
\end{align*}
Teniendo en cuenta esto, es natural definir la familia de operadores $\{U(t)\}_{t\geq 0}$ tal que
$$U(t)f(x)=\begin{cases}
    \displaystyle\sum_{k\in\Z^n}e^{(ik_1|k|^2-|k|^2)t+ik\cdot x}\widehat{f}(k), & t>0,\\
    f(x), &t=0
\end{cases}$$
Para $f$ lo suficientemente regular.\\
 A continuación presentamos una propiedad de estos operadores:
\begin{prop}\label{FourierUt}
    Si $f\in L^2(\T^n)$ entonces $U(t)f\in L^2(\T^n),$ ademas
    $$\hat{U(t)f}(k)=e^{(ik_1|k|^2-|k|^2)t}\hat{f}(k).$$
    Para todo $k\in \Z^n.$ 
 \end{prop} 


De esta manera, nuestro candidato a solución es la fórmula de Duhamel dada por
\begin{equation}\label{Duhamel}
u(x,t)=U(t)u_0(x)-\int_0^tU(t-t^\prime)u^3(x,t^\prime)\,dt^\prime,
\end{equation}
\begin{note}
En algunas ocasiones omitiremos la dependencia espacial con el propósito de no sobrecargar la notación.
\end{note}
\begin{definition}
    Dado $s\geq 0.$ Diremos que el problema (\ref{ZKB}) está localmente bien planteado en $H^s(\T^n)$ si:
    \begin{itemize}
        \item \textbf{(Existencia y Unicidad).} Dado $u_0\in H^s(\T^n),$ existe $T>0$ y una única solución de la fórmula de Duhamel (\ref{Duhamel}) con dato inicial $u_0$ en el espacio $C([0,T];H^s(\T^n)).$
        \item \textbf{(Dependencia Continua).} Dado $u_0\in H^s(\T^n)$ existe una vecindad $V$ de $u_0$ en $H^s(\T^n)$ y $T>0$ tal que la aplicación dato inicial solución 
        $$v_0\in H^s(\T^n)\mapsto v\in C([0,T];H^s(\T^n)), $$ 
        es continua.
    \end{itemize}
\end{definition}
Antes de continuar con el resultado principal debemos hacer varias salvedades. Primero definimos el espacio $C([0,T];H^s(\T^n))$.
\begin{definition}
     El espacio $C([0,T];H^s(\T^n))$. Es el conjunto de funciones $f:[0,T]\to H^s(\T^n)$, continuas en el siguiente sentido:\\ 
 Dado $t^\prime\in(0,T)$ tenemos que
$$\lim_{t\to t^\prime}\|f(t)-f(t^\prime)\|_{H^s}=0,$$
si $t^\prime=0$ o $t^\prime=T$ se toma el limite lateral.
 \end{definition}
Una propiedad importante de este espacio es
\begin{prop}\label{completezC}
    El espacio $C([0,T];H^s(\T^n))$ es completo con la distancia
    $$\sup_{t\in[0,T]}\|u(t)-v(t)\|_{H^s}.$$
    Para $u,v\in C([0,T];H^s(\T^n)).$
\end{prop}
\begin{proof}
    Sea $\{f_l\}_{l\in\Z^+}$ una sucesión de Cauchy en $C([0,T];H^s(\T^n))$, así, para todo $\epsilon>0$ existe $N\in\Z^+$, tal que si $i,j\geq N$,  entonces, $\sup_{t\in[0,T]}\|f_i(t)-f_j(t)\|_{H^s}<\epsilon,$ luego, para $t\in[0,T]$ fijo tenemos que $\|f_i(t)-f_j(t)\|_{H^s}<\epsilon,$ es decir $\{f_l(t)\}_{l\in\Z^+}$ es una sucesión de Cauchy en $H^s(\T^n)$. Por el teorema (2.13), esta sucesión converge de manera puntual a alguna función $f(t)\in H^s(\T^n).$ Primero probemos que $f_l\to f$ en $C([0,T];H^s(\T^n)).$\\

    Sea $t\in[0,T]$ fijo pero arbitrario, como $\{f_l\}_{l\in\Z^+}$ es una sucesión de Cauchy en $C([0,T];H^s(\T^n))$, existe $N\in\Z^+$, tal que si $i,j\geq N$, entonces $\sup_{t\in[0,T]}\|f_i(t)-f_j(t)\|_{H^s}<\frac{\epsilon}{2},$ como tenemos convergencia puntual en $H^s(\T^n)$, existe $K(t)\in\Z^+,$ tal que si $l\geq K(t),$ entonces $\|f_l(t)-f(t)\|_{H^s}<\frac{\epsilon}{2}.$ Si tomamos $i\geq N$ y $j=\max\{N,K(t)\}$ 
    \begin{align*}
        \|f_i(t)-f(t)\|_{H^s}&=\|f_i(t)-f_j(t)+f_j(t)-f(t)\|_{H^s}\\
        &\leq\|f_i(t)-f_j(t)\|_{H^s}+\|f_j(t)-f(t)\|_{H^s}\\
        &\leq\sup_{t\in[0,T]}\|f_i(t)-f_j(t)\|_{H^s}+\|f_j(t)-f(t)\|_{H^s}\\
        &<\frac{\epsilon}{2}+\frac{\epsilon}{2}=\epsilon.
    \end{align*}
    Esto se tiene para $t\in[0,T]$ arbitrario, y para todo $i\geq N$, entonces $\sup_{t\in[0,T]}\|f_i(t)-f(t)\|_{H^s}<\epsilon.$\\

    Para finalizar la prueba basta con probar que $f\in C([0,T];H^s(\T^n))$.\\ 

    Sea $t\in[0,T]$ y $t^\prime\in[0,T]$ dado. Primero, como cada $f_i\in C([0,T];H^s(\T^n))$ tenemos que para $\epsilon>0,$ existe un $\delta>0,$ tal que si $|t-t^\prime|<\delta$, entonces $\|f_i(t)-f_i(t^\prime)\|<\frac{\epsilon}{3},$ y por lo probado anteriormente sabemos que existe $N\in\Z^+$ tal que si $i\geq N$, entonces $\sup_{t\in[0,T]}\|f(t)-f_i(t)\|_{H^s}<\frac{\epsilon}{3}.$ Con todo esto notamos que
    \begin{align*}
        \|f(t)-f(t^\prime)\|_{H^s}&\leq\|f(t)-f_i(t)\|_{H^s}+\|f_i(t)-f_i(t^\prime)\|_{H^s}+\|f_i(t^\prime)-f(t^\prime)\|_{H^s}\\
        &\leq 2\sup_{t\in[0,T]}\|f(t)-f_i(t)\|_{H^s}+\|f_i(t^\prime)-f(t^\prime)\|_{H^s}\\
        &<\frac{2}{3}\epsilon+\frac{\epsilon}{3}=\epsilon.
    \end{align*}
    Mostrando así, que para $t^\prime\in[0,T]$ arbitrario tenemos que
    $$\lim_{t\to t^\prime}\|f(t)-f(t^\prime)\|_{H^s}=0.$$
\end{proof}
Segundo, introducimos los conceptos claves de contracción y punto fijo. 
\begin{definition}Sea $(M,d)$ un espacio métrico y $\psi:M\to M$ una función.
    \begin{itemize}
        \item $\psi$ es una \textbf{contracción} si existe $0<L<1$ tal que
        $$d(\psi(x),\psi(y))\leq Ld(x,y),$$
        para todo $x,y\in M.$
        \item Dado $x\in M,$ si $\psi(x)=x$, se dice que $x$ es un \textbf{punto fijo} de $\psi.$
    \end{itemize}
\end{definition}
\begin{theorem}[Punto fijo de Banach]\label{Banach}
Sea $(M,d)$ un espacio métrico completo. entonces toda contracción tiene un único punto fijo.   
\end{theorem}
La idea para el buen planteamiento es ver (\ref{Duhamel}) como una aplicación y usar el teorema (\ref{Banach}) para encontrar soluciones como puntos fijos en espacios de funciones adecuados. Con todo esto procedemos a la demostración de el resultado principal, la buena colocación local.
\begin{theorem}
    Para cualquier $n\geq 1$ fijo, el problema de Cauchy (\ref{ZKB}) está localmente bien planteado en $H^s(\T^n),$ $s>\frac{n}{2}.$ Esto es:
    \begin{itemize}
        \item \textbf{(Existencia y Unicidad).} Para cualquier $u_0\in H^s(\T^n),$ existe un tiempo $T>0$ y una única $u\in C([0,T];H^s(\T^n))$ solución de la fórmula integral (\ref{Duhamel}) con dato inicial $u_0.$
        \item \textbf{(Dependencia Continua).} Dado $u_0\in H^s(\T^n)$ existe una vecindad $V$ de $u_0$ en $H^s(\T^n)$ y $T>0$ tal que la aplicación dato inicial solución 
        $$v_0\in V\mapsto v\in C([0,T];H^s(\T^n)), $$ 
        es continua. 
    \end{itemize}
\end{theorem}
\begin{proof}
Dividiremos la prueba en tres partes.
\begin{itemize}
       \item[i)]\textbf{Existencia.} Sea $u_0\in H^s(\T^n)$ con $s>\frac{n}{2}$ arbitrario pero fijo. Veamos que existe una solución de (\ref{Duhamel}) con dato inicial $u_0.$\\
        Si $u_0=0,$ tomando $u=0$ tenemos la existencia. Ahora si asumimos $u_0\neq 0$, tenemos que $\|u_0\|_{H^s}>0.$ Dados $T>0$ y $a>0$ fijos definimos el espacio
       $$X_T^s(a)=\{v\in C([0,T];H^s(\T^n)):\sup_{t\in[0,T]}\|v(t)\|_{H^s}\leq a\}.$$
       Primero, probemos que $X_T^s(a)$ es completo con la distancia
       $$\sup_{t\in[0,T]}\|u(t)-v(t)\|_{H^s},$$
       para $u,v\in X^s_T(a)$.\\

       Note que $X^s_T(a)$ es un subconjunto cerrado de $C([0,T];H^s(\T^n))$, ya que es la bola cerrada de centro $0$ en ese espacio. Por la proposición (\ref{completezC}) y el hecho de que todo subconjunto cerrado de un espacio métrico completo es completo, concluimos que $X^s_T(a)$ es completo bajo esa distancia.

       Con esto, sea $v\in X^s_T(a)$, definimos la función
       $$\phi(v)(x,t)=U(t)u_0(x)-\int_0^tU(t-t^\prime)v^3(x,t^\prime)\,dt^\prime.$$
       Veamos que $\phi(v)\in X^s_T(a).$ 

       Utilizando la definición de la norma en $H^s(\T^n)$ y la proposición (\ref{FourierUt}) tenemos que
       \begin{align*}
           \|\phi(v)(x,t)\|_{H^s}&\leq\|U(t)u_0(x)\|_{H_s}+\int_0^t\|U(t-t^\prime)v^3(x,t^\prime)\|_{H^s}\,dt^\prime\\
           &=\|\langle k\rangle^s\widehat{U(t)u}_0(k)\|_2+\int_0^t\|\langle k\rangle^s(U(t-t^\prime)v^3)^\wedge(k,t^\prime)\|_2\,dt^\prime\\
           &=\|e^{(ik_1|k|^2-|k|^2)t}\langle k\rangle^s\widehat{u}_0(k)\|_2+\int_0^t\|e^{(ik_1|k|^2-|k|^2)(t-t^\prime)}\langle k\rangle^s\widehat{v^3}(k,t^\prime)\|_2\,dt^\prime\\
           &\leq\|u_0(x)\|_{H^s}+\int_0^t\|v^3(x,t^\prime)\|_{H^s}\,dt^\prime\\
           &\leq\|u_0(x)\|_{H^s}+c\int_0^t\|v(x,t^\prime)\|^3_{H^s}\,dt^\prime.
       \end{align*}
       Note que esto lo podemos hacer ya que $H^s(\T^n)$ es álgebra de Banach y 
       $$\left|e^{(ik_1|k|^2-|k|^2)t}\right|=\left|e^{ik_1|k|^2t}\right|\left|e^{-|k|^2t}\right|\leq 1.$$
       Si tomamos el supremo a ambos lados de la desigualdad tenemos que
       \begin{align*}
           \sup_{t\in[0,T]}\|\phi(v)(x,t)\|_{H^s}&\leq\|u_0(x)\|_{H^s}+cT\sup_{t\in[0,T]}\|v(x,t)\|^3\\
           &\leq\|u_0(x)\|_{H^s}+cTa^3,
       \end{align*}
       esto último ya que $v\in X_T^s(a).$ Si escogemos
       $$\begin{cases}
           a=\dfrac{3}{2}\|u_0(x)\|_{H^s},\\
           0<T<\dfrac{4}{27c}\|u_0(x)\|^{-2}_{H^s},
       \end{cases}$$
       tenemos que
       $$\sup_{t\in[0,T]}\|\phi(v)(x,t)\|_{H^s}<\|u_0(x)\|_{H^s}+c\left(\dfrac{4}{27c}\|u_0(x)\|^{-2}_{H^s}\right)\left(\dfrac{3}{2}\|u_0(x)\|_{H^s}\right)^3 =\dfrac{3}{2}\|u_0(x)\|_{H^s}.$$
       Ahora falta probar que $\phi(v)\in C([0,T];H^s(\T^n))$. Dado $s\in[0,T]$
       \begin{align*}
           \|\phi(v)(t)-\phi(v)(s)\|_{H^s}&=\left\|U(t)u_0-\int_0^tU(t-t^\prime)v^3(t^\prime)\,dt^\prime-U(s)u_0+\int_0^sU(s-t^\prime)v^3(t^\prime)\,dt^\prime\right\|_{H^s}\\
           &\leq\|U(t)u_0-U(s)u_0\|_{H^s}+\left\|\int_0^sU(s-t^\prime)v^3(t^\prime)\,dt^\prime-\int_0^tU(t-t^\prime)v^3(t^\prime)\,dt^\prime\right\|_{H^s}.
       \end{align*}
       Luego, basta con probar que cuando $t\to s$, cada sumando se hace 0. Primero, veamos el caso donde $s=0$, note que de la definición de la norma en $H^s(\T^n)$, y por la proposición (\ref{FourierUt}) 
       \begin{align*}
           \|U(t)u_0-U(0)u_0\|_{H^s}^2&=\|\langle k\rangle^s(\hat{U(t)u}_0(k)-\hat{u_0}(k))\|_2^2\\
           &=\sum_{k\in\Z^n}|\langle k\rangle^s(e^{(ik_1|k|^2-|k|^2)t}-1)\hat{u}_0(k)|^2.
       \end{align*}
       Ahora bien,
       \begin{align*}
           |\langle k\rangle^s(e^{(ik_1|k|^2-|k|^2)t}-1)\hat{u}_0(k)|^2&\leq(|e^{(ik_1|k|^2-|k|^2)t}|+|1|)^2||\langle k\rangle^s\hat{u}_0(k)|^2\\
           &\leq 4|\langle k\rangle^s\hat{u}_0(k)|^2,
       \end{align*}
        por hipótesis $\{\langle k\rangle^s\hat{u}_0(k)\}\in\ell^2(\Z^n)$, por el criterio M de Weierstrass  $\{\langle k\rangle^s(e^{(ik_1|k|^2-|k|^2)t}-1)\hat{u}_0(k)\}\in\ell^2(\Z^n)$. Así,
       $$\lim_{t\to0^+}\|U(t)u_0-U(0)u_0\|_{H^s}=\left(\sum_{k\in\Z^n}\lim_{t\to0^+}|\langle k\rangle^s(e^{(ik_1|k|^2-|k|^2)t}-1)\hat{u}_0(k)|^2\right)^{\frac{1}{2}}=0.$$
       Ahora, para el otro sumando tenemos que
       
        \begin{align*}
             \left\|\int_0^sU(s-t^\prime)v^3(t^\prime)\,dt^\prime-\int_0^tU(t-t^\prime)v^3(t^\prime)\,dt^\prime\right\|_{H^s}&\leq\int_0^t\|U(t-t^\prime)v^3(t^\prime)\|_{H^s}\,dt^\prime\\
             &\leq\int_0^t\|v^3(t^\prime)\|_{H^s}\,dt^\prime\to 0
         \end{align*} 

       Cuando $t\to0^+$, de esta manera,

       $$\lim_{t\to0^+}\|\phi(v)(t)-\phi(v)(0)\|_{H^s}=0.$$
       Con esto concluimos el caso para $s=0$.

       Veamos el caso cuando $s\neq 0$, podemos asumir sin pérdida de generalidad que $t<s$, de manera similar para el primer sumando
       \begin{align*}
           \|U(t)u_0-U(s)u_0\|_{H^s}^2&=\|\langle k\rangle^s(\hat{U(t)u}_0(k)-\hat{U(s)u_0}(k))\|_2^2\\
           &=\sum_{k\in\Z^n}|\langle k\rangle^s(e^{(ik_1|k|^2-|k|^2)t}-e^{(ik_1|k|^2-|k|^2)s})\hat{u}_0(k)|^2.
       \end{align*}
       Luego,
       \begin{align*}
           |\langle k\rangle^s(e^{(ik_1|k|^2-|k|^2)t}-e^{(ik_1|k|^2-|k|^2)s})\hat{u}_0(k)|^2&\leq(|e^{(ik_1|k|^2-|k|^2)t}|+|e^{(ik_1|k|^2-|k|^2)s}|)^2||\langle k\rangle^s\hat{u}_0(k)|^2\\
           &\leq 4|\langle k\rangle^s\hat{u}_0(k)|^2,
       \end{align*}
       y nuevamente por el criterio M de Weierstrass podemos intercambiar el límite y la suma, es decir
       $$\lim_{t\to s}\|U(t)u_0-U(s)u_0\|_{H^s}=\left(\sum_{k\in\Z^n}\lim_{t\to s}|\langle k\rangle^s(e^{(ik_1|k|^2-|k|^2)t}-e^{(ik_1|k|^2-|k|^2)s})\hat{u}_0(k)|^2\right)^{\frac{1}{2}}=0.$$
       Para el segundo sumando tenemos

       \begin{align*}
             \left\|\int_0^sU(s-t^\prime)v^3(t^\prime)\,dt^\prime-\int_0^tU(t-t^\prime)v^3(t^\prime)\,dt^\prime\right\|_{H^s}&\leq\int_t^s\|U(s-t^\prime)v^3(t^\prime)-U(t-t^\prime)v^3(t^\prime)\|_{H^s}\,dt^\prime\\
             &\leq\int_t^s\|U(s-t^\prime)v^3(t^\prime)\|_{H^s}+\|U(t-t^\prime)v^3(t^\prime)\|_{H^s}\,dt^\prime\\
             &\leq2\int_t^s\|v^3(t^\prime)\|_{H^s}dt^\prime\to 0.
         \end{align*} 
         Cuando $t\to s$. Es decir
         $$\lim_{t\to s}\|\phi(v)(t)-\phi(v)(s)\|_{H^s}=0.$$
       Con estos dos hechos podemos concluir que $\phi:X^s_T(a)\to X^s_T(a).$ Ahora, veamos que $\phi$ define una contracción para los $a$ y $T$  previamente dados. Sean $u,v\in X^s_T(a),$  por la definición de $\phi$ y de manera similar al argumento previo tenemos
       \begin{align*}
           \|\phi(v)(t)-\phi(u)(t)\|_{H^s}&=\left\|-\int_0^tU(t-t^\prime)v^3(t^\prime)\,dt^\prime+\int_0^tU(t-t^\prime)u^3(t^\prime)\,dt^\prime\right\|_{H^s}\\
           &\leq\int_0^t\|U(t-t^\prime)(u^3(t^\prime)-v^3(t^\prime))\|_{H^s}\,dt^\prime\\
           &\leq\int_0^t\|u^3(t^\prime)-v^3(t^\prime)\|_{H^s}\,dt^\prime\\
           &=\int_0^t\|(u-v)(u^2+uv+v^2)(t^\prime)\|_{H^s}\,dt^\prime\\
           &\leq c\int_0^t\|u(t^\prime)-v(t^\prime)\|_{H^s}\|(u^2+uv+v^2)(t^\prime)\|_{H^s}\,dt^\prime\\
           &\leq c\int_0^t\|u(t^\prime)-v(t^\prime)\|_{H^s}(\|u(t^\prime)\|^2_{H^s}+\|u(t^\prime)\|_{H^s}\|v(t^\prime)\|_{H^s}+\|v(t^\prime)\|^2_{H^s})\,dt^\prime.
       \end{align*}

        Con esto si tomamos el supremo a ambos lados respecto a $t$ tenemos que
       $$\sup_{t\in[0,T]}\|\phi(v)(x,t)-\phi(u)(x,t)\|_{H^s}\leq 3ca^2T\sup_{t\in[0,T]}\|u(x,t)-v(x,t)\|_{H^s}.$$
       Luego,
       $$0<3ca^2T<3c\left(\dfrac{3}{2}\|u_0(x)\|_{H^s}\right)^2\left(\dfrac{4}{27c}\|u_0(x)\|^{-2}_{H^s}\right)=1$$
       De esta manera concluimos que $\phi$ es una contracción sobre un espacio métrico completo, y por el teorema (\ref{Banach}) tenemos que existe una única función $u\in X^s_T(a)$ donde $\phi(u)=u,$ esto quiere decir que $u$ es una solución de (\ref{Duhamel}). Finalizando así la prueba de  existencia.
       \item \textbf{Unicidad.} Sean $u,v\in C([0,T];H^s(\T^n))$ soluciones de (\ref{Duhamel}), luego existe $M>0$ tal que
       $$\sup_{t\in[0,T]}\|u(t)\|_{H^s}+\sup_{t\in[0,T]}\|v(t)\|_{H^s}\leq M.$$
       Así, siguiendo un argumento similar al ítem anterior, tenemos
       \begin{align*}
           \|u(t)-v(t)\|_{H^s}&=\left\|\int_0^tU(t-t^\prime)v^3(t^\prime)\,dt^\prime-\int_0^tU(t-t^\prime)u^3(t^\prime)\,dt^\prime\right\|_{H^s}\\
           &\leq\int_0^t\|U(t-t^\prime)(v^3(t^\prime)-u^3(t^\prime))\|_{H^s}\,dt^\prime\\
           &\leq\int_0^t\|v^3(t^\prime)-u^3(t^\prime)\,dt^\prime\|_{H^s}\\
           &\leq c\int_0^t\|v(t^\prime)-u(t^\prime)\|_{H^s}(\|u(t^\prime)\|^2_{H^s}+\|u(t^\prime)\|_{H^s}\|v(t^\prime)\|_{H^s}+\|v(t^\prime)\|^2_{H^s})\,dt^\prime\\
           &\leq c\int_0^t\|v(t^\prime)-u(t^\prime)\|_{H^s}(\|u(t^\prime)\|_{H^s}+\|v(t^\prime)\|_{H^s})^2\,dt^\prime
       \end{align*}
       Luego tomando el supremo entre 0 y $T_1$, donde $0<T_1\leq T$, tenemos
       $$\sup_{t\in[0,T^1]}\|u(t)-v(t)\|_{H^s}\leq cT_1M^2\sup_{t\in[0,T_1]}\|v(t)-u(t)\|_{H^s}. $$
       Si escogemos $T_1=\min\{T,\frac{1}{2cM^2}\}$, junto a la desigualdad anterior, obtenemos
       $$\sup_{t\in[0,T^1]}\|u(t)-v(t)\|_{H^s}\leq \frac{1}{2}\sup_{t\in[0,T_1]}\|v(t)-u(t)\|_{H^s}.$$
       Note que esto solo es posible si $u(t)=v(t)$ para todo $t\in[0,T_1].$ Si $T_1=T$ hemos finalizado la prueba, ya que hemos mostrado que $u(t)=v(t)$ para todo $t\in[0,T].$ En caso contrario, es decir, si $T_1<T$, como $c$ es una constante positiva que no depende de $T$ ni de $u$ o $v$, de hecho, solo depende de la constante universal que dada en el teorema (\ref{algebra}). Ademas como $M$ es fijo, por la primera parte podemos plantear el problema de valor inicial (\ref{ZKB}) con la condición inicial $u(x,T_1)=v(x,T_1)=u_0(x)$ y de la misma manera, deducir que $u(t)=v(t)$ para todo $t\in[T_1,T_2]$, donde $T_2=\min\{T,2T_1\},$ implicando así que $u(t)=v(t)$ para todo $t\in[0,T_2],$ este argumento lo podemos iterar para todo $k\in \Z^+.$ Eventualmente existirá un $k$ tal que $kT_1\geq T$, es decir que $T_k=T,$ y por tanto en la iteración $k$-esima tendremos que $u(t)=v(t)$ para todo $t\in[0,T],$ concluyendo así la unicidad.
       \item \textbf{Dependencia Continua.}


   \end{itemize}   
\end{proof}



\newpage
\bibliographystyle{unsrt}

\bibliography{references}
\nocite{*}


\end{document}


